\section{Einleitung}\label{sec:einleitung}

Ärzte und Zahnärzte haben den Anspruch in Ihren Praxen ein Rufsystem einzusetzen.
Dieses Rufsystem ermöglicht, dass der behandelnde Arzt über einen Knopfdruck Hilfe anfordern oder Behandlungsmaterial bestellen kann.
Zusätzlich bieten die meisten Rufsysteme die Möglichkeit eine Gegensprechfunktion zu integrieren.
Eine vom Kunden durchgeführte Marktanalyse hat gezeigt, dass die meisten auf dem Markt kommerziell erhältlichen Rufsysteme auf proprietären Standards beruhen und ein veraltetes Bussystem oder analoge Funktechnologie zur Signalübermittlung einsetzen.
Weiter können diese Systeme nicht in ein TCP/IP-Netzwerk integriert werden und über eine API extern angesteuert werden.

Im Rahmen dieser Arbeit soll ein Cloudbasiertes Praxisrufsystem entwickelt werden.
Pro Behandlungszimmer wird ein Android oder IOS basiertes Tablet installiert.
Auf diese Tablets kann die zu entwickelnde App installiert und betrieben werden.
Das Projekt deckt dabei die folgenden Ziele ab:

\begin{itemize}
    \item Definition und Entwicklung einer skalierbaren Softwarearchitektur
    \item Die App ist auf Android und IOS basierten Tablets einsetzbar
    \item Die App besitzt eine integrierte Gegesprechfunktion (1:1 oder 1:m Kommunikaion)
    \item Auf der App können beliebige Buttons konfiguriert werden, die anschliessend auf den anderen Tablets einen Alarm oder eine Meldung (Text- oder Sprachmeldung) generieren
    \item Textnachrichten können als Sprachnachricht (Text to Speech) ausgegeben werden
    \item Verschlüsselte Übertragung aller Meldungen zwischen den einzelnen Stationen
    \item Integration der Lösung in den Zahnarztbehandlungsstuhl über einen Raspberry PI
    \item Offene API für Integration in Praxisadministrationssystem
\end{itemize}

Die Hauptproblemstellung dieser Arbeit ist die sichere und effiziente Übertragung von Sprach- und Textmeldungen zwischen den einzelnen Tablets.
Dabei soll es möglich sein, dass die App einen Unicast, Broadcast und Mutlicast Übertragung der Daten ermöglicht.
Über eine offene Systemarchitektur müssen die Kommunikationsbuttons in der App frei konfiguriert und parametrisiert werden können.\cite{aufgabenstellung}\footnote{Ausgangslage, Ziele und Problemstellung im Originaltext der Aufabgenstellung}

Bei Projektstart bestehen zwei potenzielle Gefahren.
Keiner der Proejktteilnehmenden hat vorgängige Erfahrung mit der Entwicklung von mobilen Applikationen und der Verwendung von Amazon Webservices.
Dementsprechend ist mit Mehraufwand in diesen Bereichen zu rechnen.
Weiter besteht eine Gefahr in der pandemischen Situation bei Projektstart im Frühling 2021.
Die Organisation und Kommunikation des Projektes wird auf die Einschränkungen der aktuellen Lage angepasst und von Anfang ausschliesslich über digitale Werkzeuge organisiert.

Diese Projektarbeit soll in vier Phasen umgesetzt werden.
In der ersten Phase werden die grundlegenden Anforderungen als Meilensteine besprochen und priorisiert.
Dazu werden oberflächliche Konzepte erarbeitet.
In der zweiten Phase werden Technologien evaluiert, die verwendet werden, um die wichtigsten Anforderungen umzusetzen.
Es wird ein Proof Of Concept implementiert, um sicherzustellen, dass die technischen Voraussetzungen gegeben sind alle Anforderungen umzusetzen.
In der dritten Phase wird auf Basis des Proof Of Concept das Praxisrufsystem ``Praxisruf'' umgesetzt.
Die detaillierten Anforderungen werden dabei in einem iterativen Verfahren zusammen mit dem Kunden erarbeitet.

Im nachfolgenden Hauptteil werden die erarbeiteten Konzepte und Resultate vorgestellt.
Zuerst werden Vorgehensweise, Projektplan und die Organisation für das Projekt.
Anschliessend werden die Anforderungen vorgestellt, welche für Praxisruf umgesetzt werden.
Es wird darauf beschrieben, welche Technologien für die Umsetzung verwendet wurden und begründet, wieso diese Technologien gewählt wurden.
Das Kapitel Konzept beschreibt das detaillierte technische Konzept für Funktionsweise und Architektur von Praxisruf.
Darauf wird das umgesetzte System und Herausforderungen während der Umsetzung vorgestellt.
Am Ende der Arbeit stehen ein Fazit und Schlusswort mit Empfehlungen für das weitere Vorgehen.

\clearpage
