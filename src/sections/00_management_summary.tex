\section*{Management Summary}

Ärzte und Zahnärzte haben den Anspruch in Ihren Praxen ein Rufsystem einzusetzen.
Dieses Rufsystem ermöglicht, dass der behandelnde Arzt über einen Knopfdruck Hilfe anfordern oder Behandlungsmaterial bestellen kann.
Heute kommerziell erhältlichen Rufsysteme beruhen meistens auf proprietären Standards und veralteten Technologien. \footnote{TODO: Add Citation!}
Die Problemstellung für dieses Projekt besteht darin, ein cloudbasiertes, leicht konfigurierbares und erweiterbares Rufsystem umzusetzen.

In diesem Projekt wurde ein Konzept für ein Rufsystem zum Versenden von Benachrichtigungen erarbeitet und umgesetzt.
Das umgesetzte Praxisrufsystem wird vom Endbenutzer auf einer Mobilen Applikation benutzt.
Das Rufsystem kann von einem Administrator über eine Web-Oberfläche konfiguriert werden.
Dabei kann der Administrator pro Endgerät festlegen, welche Benachrichtigungen versendet und empfangen werden sollen.
Nach der Konfiguration kann der Endbenutzer in der Praxis das Rufsystem auf einem Tablet benutzen.
Nicht umgesetzt wurden die Funktionen Gegensprechanlage und Text To Speech für Benachrichtigungen .
Die Kommunikation zwischen den Endgeräten funktioniert ausschliesslich über konfigurierbare Benachrichtigungen.

Für das weitere Vorgehen sehen wir die folgenden 3 Optionen:


\textbf{1. Praxisruf einsetzten}

Das umgesetzte Rufsystem unterstützt das konfigurieren, versenden und Empfangen von Benachrichtigungen.
Zu diesem Zweck könnte es bereits umgesetzt werden.
Wir raten aber von dieser Option ab.
Bestehende Rufsysteme bieten of auch eine Funktion als Gegensprechanalge.
Diese Funktion fehlt dem hier umgesetzten System noch.
Weiter erlaubt die Konfiguration von Nachrichtenempfang erst sehr einfache Regeln.
Zur besseren konfigurierbarkeit sollte dieses Regelwerk erst noch erweitert werden.

\textbf{2. Praxisruf weiterentwickeln}

Das Praxisrufsystem könnte um die Funktionen Gegensprechanlage und Text To Speech erweitert werden.
Weiter könnten die verfügbaren Regeln zum Empfangen von Benachrichtigungen ausgebaut werden.
Da das System bewusst konzipiert wurde, leicht erweiterbar zu sein, können diese Änderungen eingebaut werden.
Eine Gefahr dabei ist aber die bestehende Mobile Applikation.
Unsere Erfahrung mit der dafür gewählten Technologie zeigt, dass viele Funktionen nicht für alle Plattformen funktionieren
oder Einschränkungen bezüglich Kompatibilität bringen.

\textbf{3. Praxisruf weiterentwickeln (nativ) }

Das Praxisrufsystem könne wie in Option 2 beschrieben weiterentwickelt werden.
Um Probleme bei der Mobile Applikation zu minimieren, sollte diese als native Applikation neu entwickelt werden.
Nach dem initialen Aufwand der Migration würde dies die Applikation deutlich zukunftssicherer machen.
Der Wartungsaufwand wird nach unserer Einschätzung dadurch nicht wachsen.
Der zusätzliche Aufwand zwei Applikationen zu warten wird dadurch aufgehoben, dass die Anbdingung von Mikrophon, Lautsprecher
und Gerätebenachrichtigungen deutlich besser unterstützt ist.

\textbf{Empfehlung}

Wir empfehlen deshalb mit der dritten Option fortzufahren.
Der Mobile Client sollte neu als native Applikation umgesetzt werden um bessere Kompatibilität zu gewähren.
Das Praxisrufsystem sollte zudem um die fehlenden Funktionen erweitert und weitere Konfigurationsoptionen erweitert werden.
Sobald das Praxisrufsystem auch diese Funktion unterstützt bietet es denselben Funktionsumfang wie bestehende Lösungen.
Dabei ist es allerdings leicht konfigurierbar und skalierbar und kann auf die individuellen Bedürfnisse eines Kunden zugeschnitten werden.

\clearpage
