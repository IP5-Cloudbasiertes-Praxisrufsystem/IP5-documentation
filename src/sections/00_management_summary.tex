\section*{Management Summary}

Ärzte und Zahnärzte haben den Anspruch in Ihren Praxen ein Rufsystem einzusetzen.
Dieses Rufsystem ermöglicht, dass der behandelnde Arzt über einen Knopfdruck Hilfe anfordern oder Behandlungsmaterial bestellen kann.
Heute kommerziell erhältlichen Rufsysteme beruhen meistens auf proprietären Standards und veralteten Technologien.\cite{aufgabenstellung}
Die Problemstellung für dieses Projekt besteht darin, ein cloudbasiertes, leicht konfigurierbares und erweiterbares Rufsystem umzusetzen.

In diesem Projekt wurde ein Konzept für das Praxisrufsystem ``Praxisruf'' erarbeitet und umgesetzt.
Praxisruf ist ein Rufsystem, welches für das Versenden von Benachrichtigungen in einer Praxis verwendet werden kann.
Das umgesetzte System wird von den Endbenutzern mit einer App auf einem Android Tablet oder IPad benutzt.
Die dazu entwickelte Anwendung erlaubt es den Benutzern vorkonfigurierte Benachrichtigungen zu versenden und empfangen.
Welche Benachrichtigungen versendet und empfangen werden, kann von Administratoren über eine Web-Oberfläche konfiguriert werden.
Der Funktionsumfang des umgesetzten Rufsystems beschränkt sich auch das Versenden und Empfangen von Benachrichtigungen.
Die Funktionen Gegensprechanlage und Text To Speech für Benachrichtigung wurden im Rahmen dieses Projekts nicht behandelt.

Nach Projektabschluss sehen wir nun drei Optionen für das weitere Vorgehen:


\textbf{1. Praxisruf einsetzten}

Praxisruf könnte bereits heute als produktives Benachrichtigungssystem eingesetzt werden.
Im Gegensatz zu vielen bestehenden kann Praxisruf allerdings noch nicht als Gegensprechanlage verwendet werden.
Weiter erlaubt die Konfiguration von Nachrichtenempfang erst sehr einfache Regeln.
Zur einfacheren Konfigurierbarkeit sollte dieses Regelwerk erst noch erweitert werden.
Aufgrund dieser Einschränkungen raten wir davon ab, Praxisruf in einem produktiven Umfeld einzusetzen.

\textbf{2. Praxisruf weiterentwickeln}

Praxisruf könnte um die Funktionen Gegensprechanlage und Text To Speech erweitert werden.
Weiter könnten die verfügbaren Regeln zum Empfangen von Benachrichtigungen ausgebaut werden.
Praxisruf wurde konzipiert, um leicht erweiterbar zu sein.
Dementsprechend können diese Änderungen mit verhältnissmässig geringem Aufwand eingebaut werden.
Eine mögliche Gefahr für die Weiterentwicklung ist dabei die bestehende Mobile Applikation.
Unsere Erfahrung mit der dafür gewählten Technologie zeigt, dass viele Funktionen nicht für alle Plattformen funktionieren
oder Einschränkungen bezüglich Kompatibilität bringen.
Wir raten deshalb Praxisruf weiterzuentwickeln, aber die dazugehörige App nicht in der aktuellen Form weiterzuverwenden.

\textbf{3. Praxisruf weiterentwickeln (nativ)}

Praxisruf könnte wie in Option 2 beschrieben weiterentwickelt werden.
Um Probleme bei der Appentwicklung zu minimieren, sollte dieses als native Applikation neu entwickelt werden.
Nach dem initialen Aufwand der Migration würde dies die Applikation deutlich zukunftssicherer machen.
Der Wartungsaufwand wird nach unserer Einschätzung dadurch nicht wachsen.
Der zusätzliche Aufwand zwei Applikationen zu warten wird dadurch aufgehoben, dass die Anbindung von Gerätehardware
und betriebssystemnahen Funktionen deutlich besser unterstützt ist.

\textbf{Empfehlung}

Wir empfehlen mit der Option 3 ``Praxisruf weiterentwickeln (nativ)'' fortzufahren.
Die mobile App sollte neu als native Applikation umgesetzt werden, um bessere Kompatibilität zu gewähren.
Das Praxisrufsystem sollte zudem um zusätzliche Konfigurationsoptionen und die Funktionen Gegensprechanlage erweitert werden.
Sobald Praxisruf auch diese Funktion unterstützt bietet es einen grösseren Funktionsumfang als bestehende Lösungen.
Dabei ist es aber leicht konfigurierbar und kann somit für die konkreten Bedürfnisse jedes Kunden individualisiert werden.

\clearpage
