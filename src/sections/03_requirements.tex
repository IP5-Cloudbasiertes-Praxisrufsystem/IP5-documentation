\section{Anforderungen}\label{sec:anforderungen}

Die im Rahmen des Projektes umzusetzenden Anforderungen wurden während des Projektes iterativ zusammen mit dem Kunden erarbeitet.
Dieses Kapitel zeigt den Stand der erarbeiteten Anforderungen bei Projektabschluss.
Alle Anforderungen wurden zuerst aus Fachlicher sicht mit User Stories festgehalten, die ein konkretes Bedürfnis der Benutzer beschreiben.
Weiter werden User Stories aus Sicht des Kunden festhalten, welche Rahmenbedingungen und Bedürfnisse des Auftraggebers festhalten.
Aufgrund der User Stories werden Features definiert, welche konkrete testbare Szenarien und die erwarteten Ergebnisse definieren.\footnote{Siehe Anhang E - Features und Testszenarien}

\subsubsection*{Praxismitarbeiter}

\begin{table}[h]
    \centering
    \begin{tabular}{|l|p{13cm}|c|}
        \hline
        \textbf{Id} & \textbf{Anforderung}                                                                                                                                                                                         & \textbf{Feature} \\
        \hline
        U01         & Als Praxismitarbeiter möchte ich Benachrichtigungen versenden können, damit ich andere Mitarbeiter über Probleme und Anfragen informieren kann. & F01 \\
        \hline
        U02         & Als Praxismitarbeiter möchte ich Benachrichtigungen empfangen können, damit ich auf Probleme und Anfragen anderer Mitarbeiter reagieren kann. & F02 \\
        \hline
        U03         & Als Praxismitarbeiter möchte ich nur Benachrichtigungen sehen, die für mich relevant sind, damit ich meine Arbeit effizient gestalten kann. & F02 \\
        \hline
        U04         & Als Praxismitarbeiter möchte ich über empfangene Benachrichtigungen aufmerksam gemacht werden, damit ich keine Benachrichtigungen verpasse. & F04 \\
        \hline
        U05         & Als Praxismitarbeiter möchte ich sehen welche Benachrichtigungen ich verpasst habe, damit ich auf verpasste Benachrichtigungen reagieren kann. & F04 \\
        \hline
        U06         & Als Praxismitarbeiter möchte ich eine Rückmeldung erhalten, wenn eine Benachrichtigung nicht versendet werden kann, damit Benachrichtigungen nicht verloren gehen. & F03 \\
        \hline
        U07         & Als Praxismitarbeiter möchte ich auswählen können an welchem Gerät ich das Praxisrufsystem verwende und die dafür erstellte Konfiguration erhalten, damit das Praxisrufsystem optimal verwendet werden kann. & F05 \\
        \hline
        U08          & Als Praxismitarbeiter möchte ich einen physischen Knopf am Behandlungsstuhl haben damit ich notifikationen darüber versenden kann. & F07 \\
        \hline
        U09          & Als Praxismitarbeiter möchte ich, dass mir Benachrichtigungen vorgelesen werden, damit ich informiert werde, ohne meine Arbeit unterbrechen zu müssen. & F08 \\
        \hline
        U10         & Als Praxismitarbeiter möchte ich einen anderen Client Unterhaltungen führen können damit Fragen direkt geklärt werden können. & F09 \\
        \hline
        U11         & Als Praxismitarbeiter möchte ich Unterhaltungen mit mehreren anderen Clients gleichzeitig führen können damit komplexe Fragen direkt geklärt werden können. & F10 \\
        \hline
    \end{tabular}\label{tab:userstories1}
\end{table}

\clearpage

\subsubsection*{Praxisverantwortlicher}

\begin{table}[h]
    \centering
    \begin{tabular}{|l|p{13cm}|c|}
        \hline
        \textbf{Id} & \textbf{Anforderung}                                                                                                                                                                               & \textbf{Features} \\
        \hline
        U12         & Als Praxisverantwortlicher möchte ich mehrere Geräte verwalten können, damit das Praxisrufsystem in mehreren Zimmern gleichzeitig genutzt werden kann. & F06 \\
        \hline
        U13         & Als Praxisverantwortlicher möchte ich definieren können welches Gerät, welche Anfragen versenden kann, damit jedes Gerät Verwendungsort optimiert ist. & F06 \\
        \hline
        U14         & Als Praxisverantwortlicher möchte ich die Konfiguration des Praxisrufsystems zentral verwalten können, damit das Praxisrufsystem für die Anwender optimiert werden kann. & F06 \\
        \hline
        U15         & Als Praxisverantwortlicher möchte ich definieren können, welche Geräte mit welchen anderen Geräten telefonieren können damit meinen Mitarbeitern das Arbeiten erleichtere. & F06 \\
        \hline
        U16         & Als Praxisverantwortlicher möchte ich definieren können, welche Benachrichtigung über einen physischen Knopf am Behandlungsstuhl versendet wird damit der Knopf für den Mitarbeiter optimiert ist. & F06 \\
        \hline
    \end{tabular}\label{tab:userstories2}
\end{table}

\subsubsection*{Auftraggeber}\label{subsec:auftraggeber}

\begin{table}[h]
    \centering
    \begin{tabular}{|l|p{13cm}|c|}
        \hline
        \textbf{Id} & \textbf{Anforderung}                                                                                                                                                                          & \textbf{Features} \\
        \hline
        T01         & Als Auftraggeber möchte ich, dass das Praxisrufsystem über IPads bedient werden kann, damit ich von bestehender Infrastruktur profitieren kann. & n/A \\
        \hline
        T02         & Als Auftraggeber möchte ich, dass das Praxisrufsystem über Android Tablets bedient werden kann, damit es in Zukunft für eine weitere Zielgruppe verwendet werden kann. & n/A \\
        \hline
        T03         & Als Auftraggeber möchte ich, dass die Codebasis für das Praxisrufsystem für Android und IOS verwendet werden kann, damit ich die Weiterentwicklung optimieren kann. & n/A \\
        \hline
        T04         & Als Auftraggeber möchte ich, dass wo möglich der Betrieb von Serverseitigen Dienstleistungen über AWS betrieben wird, damit ich von bestehender Infrastruktur und Erfahrung profitieren kann. & n/A               \\
        \hline
    \end{tabular}\label{tab:userstories3}
\end{table}

\clearpage

