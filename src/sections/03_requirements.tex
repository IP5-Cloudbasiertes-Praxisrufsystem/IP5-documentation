\section{Anforderungen}\label{sec:anforderungen}

Die im Rahmen des Projektes umzusetzenden Anforderungen wurden während des Projektes iterativ zusammen mit dem Kunden erarbeitet.
Dieses Kapitel zeigt den Stand der erarbeiteten Anforderungen bei Projektabschluss.
Alle Anforderungen wurden zuerst aus Fachlicher sicht mit User Stories festgehalten, die ein konkretes Bedürfnis der Benutzer beschreiben.
Weiter werden User Stories aus Sicht des Kunden festhalten, welche Rahmenbedingungen und Bedürfnisse des Auftraggebers festhalten.
Aufgrund der User Stories werden Features definiert, welche konkrete testbare Szenarien und die erwarteten Ergebnisse definieren.

\subsection{User Stories}\label{subsec:user-stories}

\subsubsection*{Praxismitarbeiter}

\begin{table}[h]
    \centering
    \begin{tabular}{|l|p{13cm}|c|}
        \hline
        \textbf{Id} & \textbf{Anforderung}                                                                                                                                                                                         & \textbf{Feature} \\
        \hline
        U01         & Als Praxismitarbeiter möchte ich Benachrichtigungen versenden können, damit ich andere Mitarbeiter über Probleme und Anfragen informieren kann. & F01 \\
        \hline
        U02         & Als Praxismitarbeiter möchte ich Benachrichtigungen empfangen können, damit ich auf Probleme und Anfragen anderer Mitarbeiter reagieren kann. & F02 \\
        \hline
        U03         & Als Praxismitarbeiter möchte ich nur Benachrichtigungen sehen, die für mich relevant sind, damit ich meine Arbeit effizient gestalten kann. & F02 \\
        \hline
        U04         & Als Praxismitarbeiter möchte ich über empfangene Benachrichtigungen aufmerksam gemacht werden, damit ich keine Benachrichtigungen verpasse. & F04 \\
        \hline
        U05         & Als Praxismitarbeiter möchte ich sehen welche Benachrichtigungen ich verpasst habe, damit ich auf verpasste Benachrichtigungen reagieren kann. & F04 \\
        \hline
        U06         & Als Praxismitarbeiter möchte ich eine Rückmeldung erhalten, wenn eine Benachrichtigung nicht versendet werden kann, damit Benachrichtigungen nicht verloren gehen. & F03 \\
        \hline
        U07         & Als Praxismitarbeiter möchte ich auswählen können an welchem Gerät ich das Praxisrufsystem verwende und die dafür erstellte Konfiguration erhalten, damit das Praxisrufsystem optimal verwendet werden kann. & F05 \\
        \hline
        U08          & Als Praxismitarbeiter möchte ich einen physischen Knopf am Behandlungsstuhl haben damit ich notifikationen darüber versenden kann. & F07 \\
        \hline
        U09          & Als Praxismitarbeiter möchte ich, dass mir Benachrichtigungen vorgelesen werden, damit ich informiert werde, ohne meine Arbeit unterbrechen zu müssen. & F08 \\
        \hline
        U10         & Als Praxismitarbeiter möchte ich einen anderen Client Unterhaltungen führen können damit Fragen direkt geklärt werden können. & F09 \\
        \hline
        U11         & Als Praxismitarbeiter möchte ich Unterhaltungen mit mehreren anderen Clients gleichzeitig führen können damit komplexe Fragen direkt geklärt werden können. & F10 \\
        \hline
    \end{tabular}\label{tab:userstories1}
\end{table}

\clearpage

\subsubsection*{Praxisverantwortlicher}

\begin{table}[h]
    \centering
    \begin{tabular}{|l|p{13cm}|c|}
        \hline
        \textbf{Id} & \textbf{Anforderung}                                                                                                                                                                               & \textbf{Features} \\
        \hline
        U12         & Als Praxisverantwortlicher möchte ich mehrere Geräte verwalten können, damit das Praxisrufsystem in mehreren Zimmern gleichzeitig genutzt werden kann. & F06 \\
        \hline
        U13         & Als Praxisverantwortlicher möchte ich definieren können welches Gerät, welche Anfragen versenden kann, damit jedes Gerät Verwendungsort optimiert ist. & F06 \\
        \hline
        U14         & Als Praxisverantwortlicher möchte ich die Konfiguration des Praxisrufsystems zentral verwalten können, damit das Praxisrufsystem für die Anwender optimiert werden kann. & F06 \\
        \hline
        U15         & Als Praxisverantwortlicher möchte ich definieren können, welche Geräte mit welchen anderen Geräten telefonieren können damit meinen Mitarbeitern das Arbeiten erleichtere. & F06 \\
        \hline
        U16         & Als Praxisverantwortlicher möchte ich definieren können, welche Benachrichtigung über einen physischen Knopf am Behandlungsstuhl versendet wird damit der Knopf für den Mitarbeiter optimiert ist. & F06 \\
        \hline
    \end{tabular}\label{tab:userstories2}
\end{table}

\subsubsection*{Auftraggeber}\label{subsec:auftraggeber}

\begin{table}[h]
    \centering
    \begin{tabular}{|l|p{13cm}|c|}
        \hline
        \textbf{Id} & \textbf{Anforderung}                                                                                                                                                                          & \textbf{Features} \\
        \hline
        T01         & Als Auftraggeber möchte ich, dass das Praxisrufsystem über IPads bedient werden kann, damit ich von bestehender Infrastruktur profitieren kann. & n/A \\
        \hline
        T02         & Als Auftraggeber möchte ich, dass das Praxisrufsystem über Android Tablets bedient werden kann, damit es in Zukunft für eine weitere Zielgruppe verwendet werden kann. & n/A \\
        \hline
        T03         & Als Auftraggeber möchte ich, dass die Codebasis für das Praxisrufsystem für Android und IOS verwendet werden kann, damit ich die Weiterentwicklung optimieren kann. & n/A \\
        \hline
        T04         & Als Auftraggeber möchte ich, dass wo möglich der Betrieb von Serverseitigen Dienstleistungen über AWS betrieben wird, damit ich von bestehender Infrastruktur und Erfahrung profitieren kann. & n/A               \\
        \hline
    \end{tabular}\label{tab:userstories3}
\end{table}

\clearpage

\subsection{Features}\label{subsec:features}

\subsubsection*{F01 - Benachrichtigungen Versenden}
\begin{tabbing}
    Left \= Middle \= Right \kill
    Scenario S01: \> \> Benachrichtigung versenden\\ \\
    Given:  \> \> Benutzer ist vollständig angemeldet\\
    And:    \> \> Mindestens 1 Empfänger ist konfiguriert\\
    When:  \> \> Praxismitarbeiter tippt auf einen Benachrichtigungs-Button\\
    Then:   \> \> Benachrichtigung wird an den zentralen Cloud Service gesendet\\
    And:    \> \> Benachrichtitung wird an alle Mobile Clients versendet\\
    \> \> die sich für diese Benachrichtigung subscribed haben weitergeleitet\\
    And:    \> \> Praxismitarbeiter erhält optische Rückmeldung, dass Benachrichtigung versendet wurde\\

    \\
    Scenario S02: \> \> Keine Empfänger konfiguriert\\ \\
    Given:  \> \>  Benutzer ist vollständig angemeldet\\
    And:  \> \>    Kein Empfänger ist konfiguriert\\
    When:  \> \>   Praxismitarbeiter tippt auf einen Benachrichtigungs-Button\\
    Then:  \> \>   Benachrichtigung wird an den zentralen Cloud Service gesendet\\
    And:  \> \>    Benachrichtitung wird nicht weitergeleitet\\

\end{tabbing}


\subsubsection*{F02 - Benachrichtigungen Empfangen}
\begin{tabbing}
    Left \= Middle \= Right \kill

    Scenario S03:  \> \> Empfangen\\ \\
    Given:  \> \>  Eine Benachrichtigung wurde von Mobile Client versendet\\
    When:  \> \>   Cloud Service Notification an Empfänger Mobile Client weiterleitet\\
    Then:  \> \>   Wird die Benachrichtigung vom Empfänger Mobile Client empfangen\\
    And:  \> \>    In einer Übersicht für empfangene Benachrichtigung angezeigt.\\

\end{tabbing}

\clearpage
\subsubsection*{F03 - Fehlgeschlagene Benachrichtigungen}

\begin{tabbing}
    Left \= Middle \= Right \kill
    Scenario S04: \> \>  Fehler Rückmeldung\\ \\
    Given: \> \>   Eine Benachrichtigung wurde von Mobile Client versendet\\
    When: \> \>    Weiterleitung von Cloud Service Notification an Empfänger schlägt auf Service Seite fehl\\
    Then: \> \>    Der Praxismitarbeiter wird über den Fehler Informiert\\
    And: \> \>     Der Praxismitarbeiter hat die Möglichkeit die Fehlgeschlagenen Benachrichtigungen zu wiederholen\\
    \\
    Scenario: \> \>  Confirm Retry\\ \\
    Given: \> \>   Benachrichtigung ist fehlgeschlagen\\
    And: \> \>     Dialog zum wiederholen wird angezeigt\\
    When: \> \>    Praxismitarbeiter bestätigt, dass wiederholt werden soll\\
    Then: \> \>    Der Cloudservice versucht erneut, die fehlgeschlagenen zuzustellen\\
    \\
    Scenario: \> \>  Cancel Retry\\ \\
    Given: \> \>   Benachrichtigung ist fehlgeschlagen\\
    And: \> \>     Dialog zum wiederholen wird angezeigt\\
    When: \> \>    Praxismitarbeiter klick, dass nicht wiederholt werden soll\\
    Then: \> \>    Werden die fehlgeschlagenen nicht wiederholt\\
    And: \> \>     Zurück zur Notificationsansicht\\

\end{tabbing}

\subsubsection*{F04 - Über Benachrichtigungen Notifizieren}
\begin{tabbing}
    Left \= Middle \= Right \kill
    Scenario S05: \> \>  Foreground\\ \\
    Given: \> \>   Mobile Client ist geöffnet\\
    When: \> \>    Eine Benachrichtigung wird vom Mobile Client emfpangen\\
    Then: \> \>    Ein Audio Signal erklingt\\
    \\
    Scenario S06: \> \>  Background\\ \\
    Given: \> \>   Mobile Client läuft im Hintergrund\\
    When: \> \>    Eine Benachrichtigung wird vom Mobile Client empfangen\\
    Then: \> \>    Ein Audio Signal erklingt\\
    And: \> \>     Eine Push Benachrichtigung wird angezeigt\\
    \\
    Scenario S07: \> \>  Nicht Quittiert\\ \\
    Given: \> \>   Mobile Client ist geöffnet\\
    And: \> \>     Eine Benachrichtigung wurde empfangen\\
    When: \> \>    Benachrichtigung wird nicht quittiert\\
    Then: \> \>    Ein Audio Signal erklingt\\
    And: \> \>     Das Audio Signal wiederholt sich alle 30 Sekunden, bis die Benachrichtigung Quittiert wurde.\\
\end{tabbing}

\clearpage

\subsubsection*{F05 - Login Mobile Client}
\begin{tabbing}
    Left \= Middle \= Right \kill
    Scenario S08: \> \>  Startbildschirm wenn nicht angemeldet\\ \\
    Given: \> \>   Mobile Client is geöffnet\\
    When: \> \>  Benutzer ist nicht angemeldet\\
    Then: \> \>  Benutzer wird zum Login aufgefordert\\
    \\
    Scenario S09: \> \>  Startbildschirm wenn angemeldet\\ \\
    Given: \> \>   Mobile Client is geöffnet\\
    When: \> \>  Benutzer ist angemeldet\\
    Then: \> \>  Konfiguration die der Benutzer zuletzt gewählt hat wird angezeigt\\
    And: \> \>    Benachrichtigungs Buttons gemäss Konfiguration werden angezeigt.\\
    \\
    Scenario S10: \> \>  Anmelden korrekt\\ \\
    Given: \> \>  Benutzer ist nicht angemeldet\\
    And: \> \>    Login Screen wird angezeigt\\
    And: \> \>     Für den Benutzer sind gültige Konfigurationen erfasst\\
    When: \> \>   Benutzer meldet sich mit korrekten Daten an\\
    Then: \> \>   Benutzer wird auf nächste Seite geleitet und kann dort die Konfiguration auswählen, die er Benutzen möchte.\\
    \\
    Scenario S11: \> \>  Anmelden falsch\\ \\
    Given: \> \>  Benutzer ist nicht angemeldet\\
    And: \> \>    Login Screen wird angezeigt\\
    When: \> \>   Benutzer meldet sich mit falschen Daten an\\
    Then: \> \>   Fehlermeldung\\
    And: \> \>  Benutzer wird nicht weitergeleitet\\
    \\
    Scenario S12: \> \>  Konfiguration Wählen\\ \\
    Given: \> \>  Benutzer hat sich korrekt angemeldet\\
    And: \> \>    Konfiguration Auswählen Screen wird angezeigt\\
    When: \> \>   Der Benutzer wählt die gewünschte Konfiguration\\
    Then: \> \>   Der Benutzer wird weitergeleitet\\
    And: \> \>    Die gewählte Konfiguration wird geladen\\
    And: \> \>    Benachrichtigungs Buttons gemäss Konfiguration werden angezeigt.\\ \\


    Scenario S13:  \> \> Logout\\ \\
    Given: \> \>  Benutzer ist angemeldet\\
    When: \> \>   Benutzer klickt logout\\
    Then: \> \>  Benutzer wird zur Login Seite weitergeleitet\\
\end{tabbing}

\clearpage

\subsubsection*{F06 - Konfigurationsverwaltung}
\begin{tabbing}
    Left \= Middle \= Right \kill
    Scenario S14: \> \>  Login\\ \\
    Given: \> \>  Benutzer ist nicht angemeldet\\
    And: \> \>    Admin UI Login Screen wird angezeigt\\
    When: \> \>   Admin meldet sich mit korrekten Daten an\\
    Then: \> \>   Admin wird auf Übersichtsseite weitergeleitet\\ \\

    Scenario S15: \> \>  Anmelden falsch\\ \\
    Given: \> \>  Benutzer ist nicht angemeldet\\
    And: \> \>    Admin UI Login Screen wird angezeigt\\
    When: \> \>   Admin meldet sich mit falschen Daten an\\
    Then: \> \>   Fehlermeldung wird angezeigt\\
    And: \> \>   Admin wird nicht weitergeleitet.\\ \\

    Scenario S16: \> \>  Konfiguration verwalten \\ \\
    Given: \> \>  Admin ist angemeldet\\
    When: \> \>  Admin UI wird aufgerufen\\
    Then: \> \>  Alle existierenden Konfigurationen werden angezeigt\\
    And: \> \>  Neue Konfigurationen können erstellt werden\\
    And: \> \>  Bestehende Konfigurationen können verändert werden\\
    And: \> \>  Bestehende Konfigurationen können gelöscht werden\\
\end{tabbing}

\subsubsection*{F07 - Integration Behandlungsstuhl}

Dieses Feature fällt ausserhalb des Projekt Scopes. Dementsprechend wurden dafür noch keine Szenarien definiert.

\subsubsection*{F08 - Text To Speech}
Dieses Feature fällt ausserhalb des Projekt Scopes. Dementsprechend wurden dafür noch keine Szenarien definiert.

\subsubsection*{F09 - Direkte Anrufe}
Dieses Feature fällt ausserhalb des Projekt Scopes. Dementsprechend wurden dafür noch keine Szenarien definiert.

\subsubsection*{F10 - Gruppen Anrufe}
Dieses Feature fällt ausserhalb des Projekt Scopes. Dementsprechend wurden dafür noch keine Szenarien definiert.



