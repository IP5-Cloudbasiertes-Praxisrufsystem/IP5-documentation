\section {Schluss}

Im Rahmen dieser Projektarbeit konnte ein cloudbasiertes Praxisrufsystem umgesetzt werden.
Das umgesetzte System besteht aus einer Mobilen Applikation für iOS und Android, einem Cloud Service und einer Web-Applikation.
Mit Firebase Messaging wurde ein externer Messaging Service angebunden.
Zudem wurde eine Entwicklungs- und Betriebsinfrastruktur mit Amazon Webservices aufgebaut, welche es erlaubt den Cloud Service und die Web-Applikation zu betreiben.

Das umgesetzte Praxisrufsystem ermöglicht es Benachrichtigungen über eine Mobile Applikation zu versenden.
Als Endgeräte können dafür IPads oder Android Tablets verwendet und empfangen werden.
Der Mobile Client ermöglicht es dabei Benachrichtigungen auch zu empfangen, wenn die Applikation im Hintergrund läuft und sammelt alle Benachrichtigungen in einer Inbox.
Welche Benachrichtigungen versendet werden können, kann über eine Web-Applikation pro Gerät konfiguriert werden.
Weiter können über die Web-Applikation Regeln definiert werden, welche Benachrichtigungen ein Gerät empfangen soll.
So ist es mit dem Praxisrufsystem möglich vorkonfigurierte Benachrichtigungen an einzelne, mehrere oder alle angebundenen Clients zu versenden.

Das umgesetzte System hat aber durchaus noch Lücken, welche eine produktive Nutzung verhindern könnten.
Das Praxisrufsystem konnte wegen diverser Herausforderungen\footnote{Siehe Kapitel 6} nicht im Umfang wie es in der Aufgabenstellung beschrieben wurde umgesetzt werden.
Die Anforderungen Benachrichtigungen, mit einer Text-To-Speech-Funktion auszugeben und eine Gegensprechanlage für Direkt- und Gruppengespräche wurden weder als Konzept noch in der Praxis umgesetzt.
Dementsprechend bietet das umgesetzte Praxisrufsystem nicht in allen Fällen den Funktionsumfang, der für eine produktive Nutzung nötig wäre.

Werden die Funktionen Text-To-Speech und Gegensprechanlage nicht benötigt, könnte das implementierte System grundsätzlich produktiv eingesetzt werden.
Mit dem aktuellen Stand des Systems ist allerdings nur die Konfiguration von einfachen Regeln möglich.
Wenn in einem produktiven System kompliziertere oder zusammengesetzte Regeln zur Vermittlung von Benachrichtigung benötigt werden, ist das Praxisrufsystem noch nicht einsetzbar.
Das System wurde allerdings so konzipiert, dass zusätzliche und kompliziertere Regeln einfach umgesetzt werden können.
Eine Erweiterung des Systems in diesem Punkt wäre deshalb mit verhältnissmässig kleinem Aufwand möglich.

Die grösste Gefahr für die produktive Nutzung des Systems ist allerdings die umgesetzte mobile Applikation.
Die Technologie mit der die Applikation umgesetzt wurde ermöglicht es, eine Codebasis für Android und iOS Geräte zu teilen.
Dies ermöglicht in einigen Bereichen schnellere Entwicklung und einfachere Wartung.
Es erschwert allerdings die Anbindung von Betriebssystemfunktionen und Zugriff auf Gerätehardeware wie es für eine Gegensprechanlage und Text-To-Speech Funktionen nötig wäre.

Insgesamt sind wir zufrieden mit den Konzepten und Ergebnissen, die aus dieser Arbeit hervorgegangen sind.
Wir sind überzeugt mit dem Konzept für die Registrierung von Mobile Clients und dem Regelwerk für Benachrichtigungen eine erweiterbare, skalierbare Grundlage geschaffen zu haben.
Das umgesetzte Rufsystem bietet eine solide Grundlage um ein konkurrenzfähiges, modernes Praxisrufsystem zu entwickeln, welches alle nötigen Anforderungen für den produktiven Einsatz abdeckt.
