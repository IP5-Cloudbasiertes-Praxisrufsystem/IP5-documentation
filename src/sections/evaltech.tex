\section{Evaluation Technologien}\label{sec:evaluation-technologien}

\subsection{Mobile Client}\label{subsec:mobile-client2}



\url{https://kotlinlang.org/lp/mobile/}
	

    +Jet Brains Infrastructure 
    +We like Kotlin 

    -IoS Env. Needed to develop for Apple 
    -Still has to develop separate API und UI Modules for Platforms 

\url{https://web.dev/progressive-web-apps/ }
	
    +No need of Native Codebase
    +Perfect for Android 
    -Eventually drawbacks because no entire API Access 
    -PWAs on IOS suck

\url{https://cordova.apache.org/} 
	

    + Popular Framework 
    + Tons of plugins to access apis 

    -Still need to have a Mac for IoS development  
    -Not a truly native app -> API Issues
 

\url{https://nativescript.org/ }

    +Provides a Workaround for nasty X-tools 
    +Claims to be truly Native 
    -Do we really trust it? (sorta new and passion project of a few people) 

 
 \url{https://flutter.dev}

    -Why do you hate me?


"Simply Write Everything twice"

    +Would definitely work

    -Do most things twice
    -We don't have time for that
    -Kunde wünscht ausdrücklich nur eine Codebasis für beide Clients.

\url{https://stackshare.io/stackups/apache-cordova-vs-nativescript}

\url{https://nativescript.org/blog/build-nativescript-apps-remotely-from-windows-or-linux/ }

\subsection{Cloud Service}\label{subsec:cloud-service2}

\url{https://aws.amazon.com/}

    \url{https://spring.io/projects/spring-boot}

    Konfig der Clients könnte sich als No-SQL anbieten.

    Config muss nur gelesen und an den Client geschickt oder abgespeichert werden

    \url{https://www.mongodb.com/}


\subsection{Betrieb und Platform}\label{subsec:betrieb-und-platform}

AWS ist MUSS