\section{Evaluation Technologien}\label{sec:evaluation-technologien}

\subsection{Mobile Client Evaluation}\label{subsec:mobile-client-eval}



\url{https://kotlinlang.org/lp/mobile/}
	

    +Jet Brains Infrastructure 
    +We like Kotlin 

    -IoS Env. Needed to develop for Apple 
    -Still has to develop separate API und UI Modules for Platforms 

\url{https://web.dev/progressive-web-apps/ }
	
    +No need of Native Codebase
    +Perfect for Android 
    -Eventually drawbacks because no entire API Access 
    -PWAs on IOS suck

\url{https://cordova.apache.org/} 
	

    + Popular Framework 
    + Tons of plugins to access apis 

    -Still need to have a Mac for IoS development  
    -Not a truly native app -> API Issues
 

\url{https://nativescript.org/ }

    +Provides a Workaround for nasty X-tools 
    +Claims to be truly Native 
    -Do we really trust it? (sorta new and passion project of a few people) 

 
 \url{https://flutter.dev}

    -Why do you hate me?


"Simply Write Everything twice"

    +Would definitely work

    -Do most things twice
    -We don't have time for that
    -Kunde wünscht ausdrücklich nur eine Codebasis für beide Clients.

\url{https://stackshare.io/stackups/apache-cordova-vs-nativescript}

\url{https://nativescript.org/blog/build-nativescript-apps-remotely-from-windows-or-linux/ }

\subsection{Messaging Service}\label{subsec:messaging-eval}

Für das Praxisrufsystem wird ein Dienst benötigt mit dem Benachrichtigungen an eine Mobile Applikation gesendet werden können.
Als Reaktion auf eine empfangene Benachrichtigung muss es möglich sein, Push-Benachrichtigungen auf dem Gerät anzuzeigen.
Die Technologie die dazu verwendet wird, muss dabei mit der für den Mobile Client gewählten Technologie kompatibel sein.
Für den Mobile Client wurde das Native Script Framework ausgewählt.
Damit sind Push-Benachrichtigungen auf iOS und Android möglich.
Push-Benachrichtigungen auf iOS sind dabei nur über die Anbindung von Firebase Messaging (FCM) möglich.\cite{nativescript-push}

Um Mobile CLients über Benachrichtigungen zu informieren ist es möglich mit EventSources\cite{event-source} Anfragen von der Serverseite an einen Client zu senden.
Dies hat allerdings den Nachteil, dass eine Konstante HTTP Verbindung zwischen Client und Server bestehen muss.
Auf Mobilen Plattformen kann nicht garantiert werden, dass diese Verbindung unbeschränkt lange offen bleibt wenn die Applikation im Hintergrund läuft.

Weiter besteht die Möglichkeit, über einen Messaging Broker Benachrichtigungen an den Mobile Client zu senden.
Hier besteht allerdings dasselbe Problem.
Damit Benachrichtigungen auf der Client Seite Empfangen werden können, muss eine Verbindung zum Messaging Broker bestehen.
Auf Mobilen Plattformen kann nicht garantiert werden, dass diese Verbindung unbeschränkt lange offen bleibt wenn die Applikation im Hintergrund läuft.

Sowohl die Option "EventSources" als auch die Option "Message Broker" haben zudem die Einschränkung, dass sie es nicht ermöglichen Push-Benachrichtigungen anzuzeigen.
Um dies auf iOS mit Nativescript zu ermöglichen müsste zusätzlich ein Firebase Messaging Service angebunden werden.

FCM ermöglicht es dabei selbst von einem Server Umfeld Benachrichtigungen an Endgeräte zu versenden.\cite{fcm-java}
Firebase Messaging unterstützt damit die Anforderungen, von einem Cloud Server Benachrichtigungen an einen Mobile Client zu senden und Push-Benachrichtigungen auf dem Gerät anzuzeigen.
In der Folge wird für das Praxisrufsystem auschliessliche Firebase Messaging als Messaging Service verwendet.

\subsection{Cloud Service}\label{subsec:cloud-service2}

Cloud Services die für das Praxisrufsystem nötig sind, werden mit Java und Spring Boot umgesetzt.
Spring Boot vereinfacht Grundlagenarbeit die zum Aufsetzen eines Backendservices nötig sind und bietet
gleichzeitig die Werkzeuge die nötig sind um sichere, skalierbare Microservices zu implementieren.\cite{why-spring}
Spring und Java sind zudem Technologien, mit welchen alle Projektteilnehmer bereits Erfahrung gesammelt haben.
Diese für die Umsetzung in diesem Projekt zu verwenden, ermöglicht es die nötigen Services noch effizienter umzusetzen
und den Fokus auf das Technische Konzept sowie die Anforderungsanalyse zu legen.

In der Anforderung T04\footnote{Siehe Kapitel 3} ist vorgegeben, dass der Betrieb aller Cloud- und Web-Applikationen mit Amazon Webservices erfolgen muss.
Mit Elastic Beanstalk von AWS ist die der Betrieb von Java Applikationen auf AWS möglich.\cite{aws-spring-java}


