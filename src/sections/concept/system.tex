\subsection{Systemarchitektur}\label{subsec:systemarchitektur}
\subsubsection*{Abgrenzung}

Dieses Kapitel gibt einen Übersicht über die Systemarchitektur als ganzes. Die Architektur beschränkt sich dabei auf die
Anforderungen die innerhalb des Projektrahmens umgesetzt wurden. Komponenten für Teile die Out Of Scope gefallen sind,
werden hier nicht behandelt.

\subsubsection*{Übersicht}

Das cloudbasierte Praxisrufsystem wird in vier Komponenten unterteilt.
Im Zentrum steht eine cloudbasierte Applikation (Cloud Service) welche es ermöglicht Konfigurationen persistent zu verwalten und das Versenden von Benachrichtigungen anhand dieser Konfigurationen koordiniert.
Der Cloud Service benutzt einen externen Messaging Service zum Versenden von Benachrichtigungen. Dabei ist der Messaging Service lediglich für die Zustellung von Benachrichtigungen verantwortlich.
Zur Verwaltung der Konfigurationen wird ein Web Frontend (Admin UI) erstellt. Dieses bietet einem Administrator die Möglichkeit Konfigurationen aus dem Cloud Service zu lesen, erstellen, bearbeiten und löschen.
Die Konfigurationen die über Admin UI und Cloud Service erstellt wurden, werden schliesslich von einem Mobile Client. Mit dem Mobile Client kann der Benutzer Benachrichtigungen an andere Mobile Clients senden.
Welche Benachrichtigungen ein Mobile Client senden kann und an wen diese Benachrichtungen zugstellt werden, wird anhand der Konfiguration aus dem Cloud Service bestimmt.

\begin{figure}[h]
    \centering
    \begin{minipage}[b]{1.0\textwidth}
        \includegraphics[width=\textwidth]{graphics/Component_System}
        \caption{System}
    \end{minipage}
\end{figure}

\clearpage

Die Anforderungen U12 setzt voraus, dass die Kommunikation im Praxisrufsystem
zwischen mehreren Geräten erfolgen kann.

Die Anforderungen U06, U07 und U13 setzten voraus, dass die es eine zentrale Stelle gibt
die Konfigurationen zur Verfügung stellt und das Versenden von Notifikationen koordiniert.


\subsubsection*{Mobile Client}

\begin{itemize}
    \item Der Mobile Client implementiert die Anbindung an den Messaging Service.
    \item Als Reaktion auf eine Notification wird eine Rückmeldung im UI angezeigt.
    \item Als Reaktion auf eine Notification wird eine OS Push Notifikation gesendet.
    Das UI bietet einen Button der eine Anfrage an die REST Schnittstelle im Cloud Service sendet.
\end{itemize}


\subsubsection*{Cloud Service}

\begin{itemize}
    \item Responsibilities (Notification and Configuration)
    \item Microservice Granularity
\end{itemize}


\subsubsection*{Messaging Service}

\begin{itemize}
    \item Dies wird ein externer Service den wir in die Applikationen einbinden. Standard hierfür ist Firebase Notifications.
    \item Der Messaging Service nimmt Notifikationen vom Cloud Service entgegen und gibt diese an den Mobile Client wieder.
    \item Dafür müssen auf beiden Seiten Komponenten eingebaut werden, die mit dem Messaging Service kommunizieren.
\end{itemize}

\clearpage
