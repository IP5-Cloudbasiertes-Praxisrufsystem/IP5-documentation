\subsection{Mobile Client}\label{subsec:mobile-client}

\subsubsection{Framework Grundlagen}
NativeScript bietet eine Abstraktion zu den nativen Plattformen Android und iOS.
Die jeweilige NativeScript Runtime erlaubt es in Javascript (oder einem entsprechenden Application Framework) Code zu schreiben,
welcher direkt für die entsprechende native Umgebung kompiliert wird~\cite{ns-core-overview}.
\begin{figure}[h]
    \centering
    \label{fig:howNSWorks}
    \includegraphics[width=0.7\textwidth]{graphics/ns-common}\caption[NativeScript-Overview]{NativeScript-Overview}\textcopyright OpenJS Foundation
\end{figure}


Die Runtime agiert als Proxy zwischen Javascript und dem jeweiligen Ökosystem.
Im Falle von iOS bedeutet dies u.A. das für alle Objective-C types ein JavaScript Prototype angeboten wird.
Dies ermöglicht es direkt mit nativen Objekten zu interagieren.
Im Umkehrschluss findet eine Typenkonversion via Marshalling Service statt\cite{ns-ios-runtime}.

\input{sections/concept/nativeScriptGuide.tex}

\subsubsection{Architektur}
\begin{figure}[h]
    \label{fig:mobileClient-packages}
    \includegraphics[width=\linewidth]{graphics/MobileClient-Architecture-export}
    \caption[Mobile-Client Package Diagramm]{Mobile-Client Package Diagramm}
\end{figure}

Der Mobile-Client wird mit modularen Komponenten aufgebaut.
Dem App-Kontext werden zwei voneinander getrennte Root-Module zur Verfügung gestellt.
Ein Modul besteht aus [1..N] Page-Modulen.
Diese Page-Module wiederum setzen sich aus eigens erstellten Komponenten und vordefinierten Komponenten des Frameworks zusammen.
Das Verhalten dieser Komponenten wird durch deren Scripts und allgemein verfügbaren Services definiert.
Der Mobile-Client wird so nach einem Objekt-orientierten Paradigma aufgebaut.

\clearpage

\begin{figure}[h]
    \label{fig:mobileClient-flow}
    \includegraphics[width=\linewidth]{graphics/MobileClient-Flow-export}
    \caption[Mobile-Client Flow Chart]{Mobile-Client Flow Chart}
\end{figure}

Das eigentliche~\emph{\nameref{fig:MobileClient-Mocks}} besteht aus dem Homescreen,
der es dem Benutzer erlaubt Nachrichten zu versenden, und der Inbox welche eingegangene Nachrichten anzeigt.
Diese Ansicht ist erst nach erfolgreicher Authentifizierung erreichbar.
Hat sich der Benutzer erfolgreich angemeldet, erhält er eine Auswahl der ihm zur verfügungstehenden Konfigurationen.
Mit der Auswahl einer dieser Konfigurationen werden die vordefinierten Notification Buttons geladen und auf dem Homescreen erstellt.

Innerhalb des App-Root Kontextes sind zwei Workflows parallel aktiv.
Zum einen können die vordefinierten Nachrichten durch Betätigen des entsprechenden Buttons versendet werden.
Die Empfänger werden durch die Rule-Engine des Cloudservices ermittelt.
Bei einem Fehlschlag wird dies dem Benutzer via Pop-Up mitgeteilt und er kann entscheiden, ob die Nachricht nochmals neu versendet werden soll.

Gleichzeitig ist immer der Listener auf Firebase aktiv.
Dieser wird auf dem Client eingegangene Nachrichten in der Inbox ablegen und ein akustisches Signal abspielen.
Wird die Meldung nicht innerhalb eines definierten Zeitraums quittiert, wird das akustische Signal wiederholt.
\clearpage

\subsubsection{User Interface}
\begin{figure}[h]
    \centering
    \begin{minipage}[b]{0.4\textwidth}
        \includegraphics[width=\textwidth]{graphics/homescreen-mockup}
        \caption{HomeScreen Mockup}
    \end{minipage}
    \hfill
    \begin{minipage}[b]{0.4\textwidth}
        \includegraphics[width=\textwidth]{graphics/mockup-received}
        \caption{Inbox Mockup}
    \end{minipage}\label{fig:MobileClient-Mocks}
\end{figure}

Die Buttons der Meldungen werden in einem 3x3 Grid auf dem Homescreen dynamisch angelegt.
Nach Betätigung eines Buttons wird dieser in Rot eingefärbt, bis eine Rückmeldung über Erfolg oder Misserfolg vom Cloudservice eingegangen ist.

Die Gegensprechfunktion ist ebenfalls auf dem Homescreen angedacht.
Hier sollten die Gesprächspartner direkt mit einem entsprechenden Button angewählt werden.
Diese Funktionalität wie beschrieben in den Userstories U10 und U11 sind im Projektverlauf ausserhalb des Umsetzungsscopes gefallen und daher nicht weiter spezifiziert worden.

Die Inbox besteht aus einer Scrollview, welche eingegangene Nachrichten als Cards darstellt.
Eine Nachricht enthält:
\begin{itemize}
    \item Titel der Nachricht
    \item Absender der Nachricht
    \item Detail Text
    \item Icon
\end{itemize}

Das Icon könnte noch dynamisch mit dem jeweiligen Nachrichtentyp verknüpft werden.
Solche wurden fachlich jedoch noch nicht definiert.

\clearpage