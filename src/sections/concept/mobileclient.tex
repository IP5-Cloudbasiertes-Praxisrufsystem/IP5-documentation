\subsection{Mobile Client}\label{subsec:mobile-client}

\subsubsection{Framework Grundlagen}
NativeScript bietet eine Abstraktion zu den nativen Plattformen Android und IOS.
Die jeweilige NativeScript Runtime erlaubt es in Javascript (oder einem entsprechenden Application Framework) Code zu schreiben,
welcher direkt für die entsprechende native Umgebung kompiliert wird~\cite{ns-core-overview}.
\begin{figure}[h]
    \centering
    \label{fig:howNSWorks}
    \includegraphics[width=0.7\textwidth]{graphics/ns-common}\caption[NativeScript-Overview]{NativeScript-Overview}\textcopyright OpenJS Foundation
\end{figure}


Die Runtime agiert als Proxy zwischen Javascript und dem jeweiligen Ökosystem.
Im Falle von IOS bedeutet dies u.A. das für alle Objective-C types ein JavaScript Prototype angeboten wird.
Dies ermöglicht es direkt mit nativen Objekten zu interagieren.
Im Umkehrschluss findet eine Typenkonversion via Marshalling Service statt\cite{ns-ios-runtime}.

\input{sections/concept/nativeScriptGuide.tex}

\subsubsection{Architektur}
\begin{figure}[h]
    \label{fig:mobileClient-packages}
    \includegraphics[width=\linewidth]{graphics/MobileClient-Architecture-export}
    \caption[Mobile-Client Package Diagramm]{Mobile-Client Package Diagramm}
\end{figure}

Der Mobile-Client wird mit modularen Komponenten aufgebaut.
Dem App-Kontext werden zwei voneinander getrennte Root-Module zur Verfügung gestellt.


\clearpage

\subsubsection{User Interface}
\begin{figure}[h]
    \centering
    \begin{minipage}[b]{0.4\textwidth}
        \includegraphics[width=\textwidth]{graphics/homescreen-mockup}
        \caption{HomeScreen Mockup}
    \end{minipage}
    \hfill
    \begin{minipage}[b]{0.4\textwidth}
        \includegraphics[width=\textwidth]{graphics/mockup-received}
        \caption{Inbox Mockup}
    \end{minipage}\label{fig:MobileClient-Mocks}
\end{figure}


\clearpage