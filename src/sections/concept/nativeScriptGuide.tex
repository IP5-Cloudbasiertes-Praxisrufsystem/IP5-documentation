\subsubsection{Anwendung}
Wir verwenden NativeScript Core als Framework des Mobile-Clients.
In Kapitel~\emph{\nameref{subsec:mobile-client-eval}} gehen wir auf die weiteren verfügbaren Frameworks ein und erläutern, weshalb wir uns gegen sie entschieden haben.

Die Client-Applikation ist in Module unterteilt.
Ein Modul wird aus folgenden Komponenten definiert:
\begin{itemize}
    \item UI-Markup: Statische Darstellung in XML
    \item Backend: Verhalten und Dynamisierung in Javascript
    \item Styling: Layout und Styles in CSS
\end{itemize}

Ein minimales Modul kann alleine aus einer XML-Datei bestehen.
Die optionalen Javascript und CSS Dateien müssen denselben Namen haben wie die XML Datei, um vom Framework korrekt verknüpft zu werden.
Dateien mit anderen Namen werden grundsätzlich vom Framework ignoriert.
Natürlich steht es Frei dennoch solche Dateien anzulegen und deren Funktionen zu verwenden z.~B. als~\emph{\nameref{subsubsec:services}} oder als~\emph{\nameref{subsubsec:code-behind-komponenten}}.

Zur Veranschaulichung der möglichen Interaktionen gehen wir auf die relevanten Aspekte des Home-Screen Modules ein.

\subsubsection*{Page Module}

\dirtree{%
    .1 app.
    .2 home.
    .3 home-page.xml.
    .3 home-page.css.
    .3 home-page.js.
    .3 home-model.js.
}

\emph{\nameref{lst:home-page.xml}} deklariert die umgebenden Komponenten.
Diese Komponenten stellen je nach Typ diverse Properties und Events zur Verfügung.
Properties können entweder statisch befüllt oder aus dem Binding-Context geladen werden.
Den Events können Callback-Functions zugewiesen werden.
Es stehen alle Funktionen zur Verfügung, welche im Backendscript~\emph{\nameref{lst:home-page.js}} exportiert werden.

\lstinputlisting[caption=home-page.xml,language=XML,label={lst:home-page.xml}]{listings/home-page.xml}

Der Binding-Context ist ein JavaScript Objekt welches exklusiv im Page-Context zur Verfügung steht.
Es ist allgemein Best-Practice dieses Objekt in einem eigenen Model zu verwalten.
Das eigentliche Binding wird vom Backendscript~\emph{\nameref{lst:home-page.js}} (Zeilen 8--11) während des ersten Ladens der Seite durchgeführt.

\lstinputlisting[caption=home-model.js,language=JavaScript,label={lst:home-model.js}]{listings/home-model.js}

Das Backendscript ist für das dynamische Verhalten der Seite verantwortlich.
Hier können die Interaktionen der Benutzer beliebig verarbeitet und der Binding-Context bei Bedarf verwaltet werden.

\lstinputlisting[caption=home-page.js,language=JavaScript,label={lst:home-page.js}]{listings/home-page.js}

\subsubsection*{Code-Behind Komponenten}\label{subsubsec:code-behind-komponenten}
Code-Behind Komponenten bieten die Möglichkeit zur Laufzeit dynamisch Grafikelemente dem UI hinzuzufügen.
Komponenten die das Framework bereits zur Verfügung stellt können direkt mit \texttt{new \textless Component\textgreater()} instanziiert werden.
Bei Bedarf können diese Komponenten auch erweitert und mit zusätzlicher Funktionalität ausgestattet werden.

Da der Home-Screen dynamisch in Abhängigkeit der Client-Configuration erstellt werden muss, werden eigene \texttt{MessageTrigger} Komponenten verwendet.
\subsubsection*{Services}\label{subsubsec:services}
In Services werden diejenigen Funktionen ausgelagert, welche nicht direkt im Zusammenhang mit der grafischen Representation stehen.
So z.~B. die REST-Calls zur API.

\clearpage