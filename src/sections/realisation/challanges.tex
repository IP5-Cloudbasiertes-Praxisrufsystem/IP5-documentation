\subsection{Herausforderungen}

\subsubsection*{Cloud Service}

Eine grosse Herausforderung war das Konzept für den Cloud Service.
Dieser musste ein konfigurierbares Subscription System bieten über den die Mobile Clients Benachrichtigungen versenden können.
Bedingung dafür war dass, individuell Regeln konfiguriert werden können die an unterschiedlichen Bedingungen prüfen, welche Benachrichtigungen für welche Clients relevant sind.
Dabei ist es wichtig, dass das Konzept es ermöglicht in Zukunft mit geringen Aufwand weitere Regeln zu implementieren.

\subsubsection*{IOS Entwicklung}

Eine weitere Herausforderung im Projekt war die Entwicklung und Konzeptierung des Mobile Clients.
Wir haben beide keine vrogägngige Erfahrung der Entwicklung von Mobilen Applikationen für Android oder IOs.
Dementsprechend mussten wir teilweise mehr Aufwand in Recherche und Entwicklung stecken als erwartet.
Hinzu kam, dass die Recherche für die zu Verwendenden Technologien im Mobile CLient sich als aufwändiger als erwartet dargestellt hat.
Dies insbesondere wegen der Bedingung, dass die gleiche Code Basis für Andorid und IOS verwendet werden kann und den Hindernissen die IOS für die meisten solchen Frameworks bietet.
Insbesondere haben es Einschränkungen für Background Aktivitäten \textbf{TODO: CITATION} und andere Barrieren im IOS Betriebssystem schwer gemacht, Messaging Services anzubinden und Push Benachrichtigungen zu versenden.
Wegen diesen Barrieren waren viele der Ansätze die wir als passend und effizient gewertet haben schlicht keine Option.

In der Umsetzung war das wenig anders.
Mit der gewählten Technologie Native Script konnten wir eine getilete Code Basis haben.
Allerdings musste die Entwicklungsumgebung schlussendlich genauso aufgesetzt werden wie es für die native Entwicklung nötig gewesen wäre.
Vom Effizienzgewinn den Nativescroipt in längeren Projekten bieten könnte, konnten wir deshalb nicht profitieren.
Die Interaktion zwischen Nativesktipt und dem IOS funktioniert über plugins. \textbf{TODO: CITATION}
Es gibt für fast alle Funktionen plugins.
Diese sind aber oft unvollständig implementiert oder funktionieren nur mit spezifischen Versionen von NativeSkript / IOS.


\subsubsection*{AWS Infrastruktur}

Ein weiterer grosser Zeitfresser war das Aufsetzen der AWS Infrastruktur.
Es war uns so schlussendlich Möglich die ganze Infrastruktur die benötigt wurde aufzusetzen.
Da wir beide wenig bis keine Erfahrung mit dem Aufsetzen von Cloud INfrastrukturen und CI/CD Pipelines haben, hat das allerdings eine Weile gedauert.
AWS bietet für alles Funktionen die wir benötigt haben ausführliche Dokumentation.
Das hat die UMsetzung deutlich vereinfacht.
Die Hürde das Ganze zuv erstehenund Umzusetzen musste trotzdem zuerst einmal genommen werden.
Insbesondere das Aufsetzen von Elastik Beanstalk mit einem Vorgestchalteten Load Balancer und HTTPS Konfiguration sowie die Anbindung an eine Relationale Datenbank hat viel Zeit gekostet.

\clearpage

\subsubsection*{API Konzept}

Zum Anfang der Konzept Pahse wurde zu wenig Zeit für das planen der API nach aussen umgesetztn.
Es wurde zuerst alles als ein einzelner Endpunkt dargestellt.
Intern gab e snur zwei Services für die Logik.
Die auszuführende Logik war insbesondere dank des Admin UIs umfassender als zuerst geplant.
Dementsprechend musste die API aufgeteilt werden, um die Weiterentwicklung und Wartbarkeit des Projektes zu gewähren.
Diese Umstellung hat allerdings wieder Zeit gebraucht.


\subsubsection*{Projektplanung}

Geplant war, dass die Milestones aus der Projektplanung als Roten Faden durch das Projekt führen.
Der agile Ansatz hat geholfen, das wir immer wussten, was als nächstes zu tun ist.
Eine leicht feingranularere Planung von Anfang an, hätte aber geholfen besser einzuorden, wie wir im Projektplan stehen.
So hätte man früher sehen können für was es nicht reicht entsprechende Massnahmen ergreifen können.

\clearpage
