\section{Umsetzung}\label{sec:umsetzung}

    \subsection{Resultate}

        Das Praxisrufsystem wurde wie im Kapitel 5 - Konzept beschrieben umgesetzt.
Es wurden die drei Komponenten Mobile Client, Cloud Service und Admin UI implementiert.
Zudem wurde als Firebase Messaging als Messaging Service angebunden, um Benachrichtigungen zwischen Mobile Clients zu versenden.
Der Mobile Client kann dabei verwendet werden, um Benachrichtigungen zu versenden und empfangen.
Cloud Service und Admin UI ermöglichen die Konfiguration für Versand und Empfang von Benachrichtigungen zu verwalten und anzuwenden.
Weiter wurde mit Amazon Web Services (AWS) eine CI/CD Umgebung aufgebaut, die es erlaubt Cloud Service und das Admin UI zu betreiben und testen.
Diese Umgebung wird dem Kunden als Template dienen, wie er das Praxisrufsystem in der Praxis betreiben kann\footnote{Siehe Anhang D}.

Im Rahmen des Projektes wurden damit die Meilensteine M01 bis M06\footnote{Siehe Kapitel 2.2} erreicht.
Umgesetzt wurden die Meilensteine mit den folgenden User Stories\footnot{Siehe Kapitel 3} inklusive aller dazu definierten Features und Szenarien\footnote{Siehe Anhang E}:

\begin{itemize}
    \item U01 - Benachrichtigung versenden
    \item U02 - Benachrichtigungen empfangen
    \item U03 - Nur relevante Benachrichtigungen empfangen
    \item U04 - Auf Benachrichtigungen aufmerksam machen
    \item U05 - Verpasste Benachrichtigungen anzeigen
    \item U06 - Fehler beim Versenden von Benachrichtigungen anzeigen
    \item U07 - Konfiguration auf Mobile Client auswählen
    \item U12 - Mehrere Mobile Clients konfigurieren
    \item U13 - Individuelle Konfiguration pro Mobile Client
    \item U14 - Zentrale Konfigurationsverwaltung
    \item T01 - Mobile Client unterstützt IPads
    \item T02 - Mobile Client unterstützt Android Tablets
    \item T03 - Geteilte Code Basis für Android and IOS
    \item T04 - Betrieb mit AWS
\end{itemize}

Die Meilensteine M06 bis M10 konnten im Rahmen dieses Projektes nicht umgesetzt werden.
Dementsprechend wurden keine Anforderungen dazu umgesetzt.
Dies betrifft die User Stories U08 bis U16.

\begin{itemize}
    \item U08 - Physicher Knopf am Behandlungsstuhl
    \item U09 - Text To Speech für Benachrichtigungen
    \item U10 - Direkte Unterhaltungen zwischen Mobile Clients
    \item U11 - Gruppenunterhaltungen zwischen Mobile Clients
    \item U15 - Konfiguration von direkten Anrufen
    \item U16 - Konfiguration von Gruppenanrufen
\end{itemize}

\clearpage

        \subsection{Mobile Client}\label{subsec:mobile-client}

\subsubsection{Framework Grundlagen}
NativeScript bietet eine Abstraktion zu den nativen Plattformen Android und iOS.
Die jeweilige NativeScript Runtime erlaubt es in Javascript (oder einem entsprechenden Application Framework) Code zu schreiben,
welcher direkt für die entsprechende native Umgebung kompiliert wird~\cite{ns-core-overview}.
\begin{figure}[h]
    \centering
    \label{fig:howNSWorks}
    \includegraphics[width=0.7\textwidth]{graphics/ns-common}\caption[NativeScript-Overview]{NativeScript-Overview}\textcopyright OpenJS Foundation
\end{figure}


Die Runtime agiert als Proxy zwischen Javascript und dem jeweiligen Ökosystem.
Im Falle von iOS bedeutet dies u.A. das für alle Objective-C types ein JavaScript Prototype angeboten wird.
Dies ermöglicht es direkt mit nativen Objekten zu interagieren.
Im Umkehrschluss findet eine Typenkonversion via Marshalling Service statt\cite{ns-ios-runtime}.

\input{sections/concept/nativeScriptGuide.tex}

\subsubsection{Architektur}
\begin{figure}[h]
    \label{fig:mobileClient-packages}
    \includegraphics[width=\linewidth]{graphics/MobileClient-Architecture-export}
    \caption[Mobile-Client Package Diagramm]{Mobile-Client Package Diagramm}
\end{figure}

Der Mobile-Client wird mit modularen Komponenten aufgebaut.
Dem App-Kontext werden zwei voneinander getrennte Root-Module zur Verfügung gestellt.
Ein Modul besteht aus [1..N] Page-Modulen.
Diese Page-Module wiederum setzen sich aus eigens erstellten Komponenten und vordefinierten Komponenten des Frameworks zusammen.
Das Verhalten dieser Komponenten wird durch deren Scripts und allgemein verfügbaren Services definiert.
Der Mobile-Client wird so nach einem Objekt-orientierten Paradigma aufgebaut.

\clearpage

\begin{figure}[h]
    \label{fig:mobileClient-flow}
    \includegraphics[width=\linewidth]{graphics/MobileClient-Flow-export}
    \caption[Mobile-Client Flow Chart]{Mobile-Client Flow Chart}
\end{figure}

Das eigentliche~\emph{\nameref{fig:MobileClient-Mocks}} besteht aus dem Homescreen,
der es dem Benutzer erlaubt Nachrichten zu versenden, und der Inbox welche eingegangene Nachrichten anzeigt.
Diese Ansicht ist erst nach erfolgreicher Authentifizierung erreichbar.
Hat sich der Benutzer erfolgreich angemeldet, erhält er eine Auswahl der ihm zur verfügungstehenden Konfigurationen.
Mit der Auswahl einer dieser Konfigurationen werden die vordefinierten Notification Buttons geladen und auf dem Homescreen erstellt.

Innerhalb des App-Root Kontextes sind zwei Workflows parallel aktiv.
Zum einen können die vordefinierten Nachrichten durch Betätigen des entsprechenden Buttons versendet werden.
Die Empfänger werden durch die Rule-Engine des Cloudservices ermittelt.
Bei einem Fehlschlag wird dies dem Benutzer via Pop-Up mitgeteilt und er kann entscheiden, ob die Nachricht nochmals neu versendet werden soll.

Gleichzeitig ist immer der Listener auf Firebase aktiv.
Dieser wird auf dem Client eingegangene Nachrichten in der Inbox ablegen und ein akustisches Signal abspielen.
Wird die Meldung nicht innerhalb eines definierten Zeitraums quittiert, wird das akustische Signal wiederholt.
\clearpage

\subsubsection{User Interface}
\begin{figure}[h]
    \centering
    \begin{minipage}[b]{0.4\textwidth}
        \includegraphics[width=\textwidth]{graphics/homescreen-mockup}
        \caption{HomeScreen Mockup}
    \end{minipage}
    \hfill
    \begin{minipage}[b]{0.4\textwidth}
        \includegraphics[width=\textwidth]{graphics/mockup-received}
        \caption{Inbox Mockup}
    \end{minipage}\label{fig:MobileClient-Mocks}
\end{figure}

Die Buttons der Meldungen werden in einem 3x3 Grid auf dem Homescreen dynamisch angelegt.
Nach Betätigung eines Buttons wird dieser in Rot eingefärbt, bis eine Rückmeldung über Erfolg oder Misserfolg vom Cloudservice eingegangen ist.

Die Gegensprechfunktion ist ebenfalls auf dem Homescreen angedacht.
Hier sollten die Gesprächspartner direkt mit einem entsprechenden Button angewählt werden.
Diese Funktionalität wie beschrieben in den Userstories U10 und U11 sind im Projektverlauf ausserhalb des Umsetzungsscopes gefallen und daher nicht weiter spezifiziert worden.

Die Inbox besteht aus einer Scrollview, welche eingegangene Nachrichten als Cards darstellt.
Eine Nachricht enthält:
\begin{itemize}
    \item Titel der Nachricht
    \item Absender der Nachricht
    \item Detail Text
    \item Icon
\end{itemize}

Das Icon könnte noch dynamisch mit dem jeweiligen Nachrichtentyp verknüpft werden.
Solche wurden fachlich jedoch noch nicht definiert.

\clearpage
        
\subsection{Cloud Service}\label{subsec:cloud-service}

\subsubsection{Architektur}

Es gibt deren Domänen 2. Configuration und Notification.

So quasi als ob man 2 Microservices haben kann. Aber wär halt doof das für den stand jetzt schon so zu trennen, deshalb vorerst mal erst ein einzelnes.

\clearpage

\subsubsection{Domänenmodell}


Für die beiden Domänen gibt es natürlich auch so n paar Diagramme. Die gibts jetzt hier:


\subsubsection*{Domäne Configuration}

\begin{figure}[h]
    \centering
    \begin{minipage}[b]{1.0\textwidth}
        \includegraphics[width=\textwidth]{graphics/Class_Configuration_Domain}
        \caption{Domänenmodell Configuration}
    \end{minipage}
\end{figure}
Also erstmal gibts da son halb generischen konfigurationsmodel.
Wär auch fast mandantenfähig.
Aber halt nur fast.



\clearpage

\begin{figure}[h]
    \centering
    \begin{minipage}[b]{1.0\textwidth}
        \includegraphics[width=\textwidth]{graphics/Class_Configuration_Services}
        \caption{Klassendiagramm Configuration Service Interfaces}
    \end{minipage}
\end{figure}
Services mit CRUD gibts hier.
Pro Service genau eine Instanz mit Default Prefix.
RulesEngine integration ist hier.
Details see Rules Engine.
Die RulesEngine sowie die RuleEvalautor Services werden ausschliesslich innerhalb der Applikation verwendet und sind nicht über die API zugreifbar.
Alle anderen Services bieten ihre Methoden über eine REST API an. Es wird darauf verzichtet, dass alles nochmal hier zu kopieren.
URL Konzept im Kapitel API.


\clearpage
Dann muss man ja noch sachen darauf auswerten können und zeug.
Das Bild zeigt: Strategy Pattern mit Spring is noch nice.

\begin{figure}[h]
    \centering
    \begin{minipage}[b]{1.0\textwidth}
        \includegraphics[width=\textwidth]{graphics/Class_Configuration_RulesEngine}
        \caption{Klassendiagramm Rules Engine}
    \end{minipage}
\end{figure}

\clearpage
\subsubsection*{Domäne Notification}

Joa, und die Notifikationen selbst muss ja auch noch wer verschicken.

\begin{figure}[h]
    \centering
    \begin{minipage}[b]{1.0\textwidth}
        \includegraphics[width=\textwidth]{graphics/Class_Notification_Domain}
        \caption{Domänenmodell Notification}
    \end{minipage}
\end{figure}

Damit man weiss was passiert und teil Zeuch wiederholt werden kann.
Braucht dann klar auch noch n paar Services.




\clearpage

\clearpage
\subsubsection{Laufzeitmodell}

Im Folgenden werden die Abläufe für das Versenden und Empfangen von Benachrichtigungen im Detail definiert.

\subsubsection*{Client Registration}

Pre-Condition: Gültige Konfiguration für Benutzer und Client sind erfasst.

In einem ersten Schritt muss sich der Praxismitarbeiter am Mobile Client anmelden.
Hat er gültige Benutzerdaten angegeben, werden Informationen zu allen verfügbaren Konfigurationen vom Cloud Service geladen und der Benutzer kann die gewünschte Konfiguration auswählen.
Dabei werden nur Name und Id der Konfigurationen geladen, damit nicht mehr Daten als nötig übertragen werden.
Sobald eine Konfiguration ausgewählt ist, werden alle dafür Konfigurierten NotificationTypes geladen und im UI die entsprechenden Buttons erstellt.
Sobald eine Konfiguration geladen ist, registriert sich der Mobile Client beim Messageing Service.
Als Antwort erhält er ein eindeutiges Token, welches verwendet werden kann, um an diesem Client Nachrichten zu senden.
Der Mobile Client Registriert sich schliesslich mit dem Token vom Messaging Service und der ausgewählten Konfiguration beim Cloud Service.
In diesem Zustand ist der Client dem Messaging Service und dem Cloud Service bekannt und ist bereit Benachrichtigungen zu empfangen.

\begin{figure}[h]
    \centering
    \begin{minipage}[b]{1.0\textwidth}
        \includegraphics[width=\textwidth]{graphics/Sequence_Notification_Register}
        \caption{Ablauf Registration}
    \end{minipage}
\end{figure}


\subsubsection*{Benachrichtigung versenden und empfangen}

Pre-Condition: Client Registrierung ist für 2 Clients abgeschlossen. Es bestehen gültige Subscriber Configs. Konfiguration ist geladen.

Der Benutzer tippt auf einen der Benachrichtigungsbuttons.
Pro Button ist die Id des verbundenen NotificationTypes hinterlegt.
Es wird nun eine Anfrage an den NotificationController im Cloud Service gesendet.
Darin enthalten sind die Id des Absender Clients und die Id des NotificationTypes.
Der NotificationController macht eine Anfrage an den ConfigurationController um alle relevanten empfänger zu finden.
Im ConfigurationController werden alle konfigurierten ClientConfigurations nach Subscriber Regeln evaluiert.
Der ConfigurationController gibt schliesslich eine Liste der Registrations zurück die zu einer ClientConfiguration gehören für die eine der Konfigurierten Regeln zugetroffen hat.
Der NotificationController lädt den NotificationType aus der Send-Anfrage und benutzt diese Daten um eine Benachrichtigung an alle Empfänger für die er gerade Registrations geladen hat zu senden.
Der NotificationController meldet zurück, ob der Versand an alle Empfänger funktioniert hat.
Ist dies nicht der Fall wird der Retry-Process auf Client Seite gestartet.

\begin{figure}[h]
    \centering
    \begin{minipage}[b]{1.0\textwidth}
        \includegraphics[width=\textwidth]{graphics/Sequence_Notification_Send}
        \caption{Ablauf Benachrichtigung Senden und Empfangen}
    \end{minipage}
\end{figure}


\clearpage
\subsubsection*{Benachrichtigung wiederholen}

Pre-Condition: Benachrichtigung versendet und mind. 1 Versand fehlgeschlagen.

Der Client zeigt einen Dialog an in dem der Benutzer informiert wird und gefragt wird, ob er die Fehlgeschlagenen wiederholen möchte.
Bestätigt der Benutzer wird eine Retry-Anfrage an den NotificationController gesendet.
Parameter ist die technische id der Notification die fehlgeschlagen ist.
NotificationController durchsucht die NotificationSendProcess tabelle nach der gegebenen id und filtert auf fehlgeschlagene.
Anschliessend wird der send prozess anhand der tokens in dieser NotificationSendProcess Instanzen wiederholt.

\begin{figure}[h]
    \centering
    \begin{minipage}[b]{1.0\textwidth}
        \includegraphics[width=\textwidth]{graphics/Sequence_Notification_Retry}
        \caption{Ablauf Benachrichtigung Wiederholen}
    \end{minipage}
\end{figure}


\clearpage

\subsubsection{API}

\subsubsection*{Verwaltung}

Um die Verwaltung der Konfigurationen zu ermöglichen, bietet der Cloud Service eine REST-API an über die Konfigurationsobjekte verwaltet werden können.
Der Cloud Service bietet Verwaltungs APIs für die Folgenden Domänenobjekte:

\begin{table}[h]
    \centering
    \begin{tabular}{|l|p{13cm}|}
        \hline
        \textbf{Domänenobjekt} & \textbf{Entity Name} \\
        \hline
        Client         & /api/clients \\
        \hline
        Client Configuration         & /api/client-configurations \\
        \hline
        Notification Types         & /api/notification-types \\
        \hline
        Users         & /api/users \\
        \hline
    \end{tabular}\label{tab:adminapimethods}
\end{table}

Für jedes dieser Domänenobjekte ist ein dedizierter Endpoint definiert, der unter dem Subpfad /api/entity-name erreichbar ist.
Jeder dieser Kontroller bietet eine API zu Verwaltung dieses Domänenobjekts, welche dem folgenden Schema folgt:

\begin{table}[h]
    \centering
    \begin{tabular}{|l|l|l|l|l|}
        \hline
        \textbf{Action} & \textbf{HTTP} & \textbf{Pfad} & \textbf{Body} & \textbf{Response} \\
        \hline
            Alle Elemente lesen         & GET & /api/entity-name & - & [EntityDto] \\
        \hline
            Einzelnes Element lesen         & GET & /api/entity-name/id & - & EntityDto \\
        \hline
            Neues Element erstellen         & POST & /api/entity-name  & EntityDto & EntityDto\\
        \hline
            Bestehendes Element ändern          & PUT & /api/entity-name  & EntityDto & EntityDto\\
        \hline
            Einzelnes Element löschen          & DELETE & /api/entity-name/id  & - & -  \\
        \hline
            Mehrere Elemente löschen          & DELETE & /api/entity-name/ids  & - & - \\
        \hline
    \end{tabular}\label{tab:apimethods}
\end{table}

Eine Ausnahme bildet hier das Domänenobjekt Registration.
Die Registrierung darf nie manuell von einem Administrator erfasst werden.
Sie muss immer von einem Mobile Client her kommen der sich beim Cloud Service registriert oder de-registriert.
Lesender Zugriff auf die Registrierungen ist nur nötig, wenn der Cloud Service eine Benachrichtigung versendet und die Messaging Tokens der relevanten Empfänger laden muss.
Dementsprechend bietet der RegistrationController nicht die volle Verwaltungs API sondern nur folgendes Subset:

\begin{table}[h]
    \centering
    \begin{tabular}{|l|l|l|l|l|}
        \hline
        \textbf{Aktion} & \textbf{HTTP} & \textbf{Pfad} & \textbf{Body} & \textbf{Response} \\
        \hline
        Registrierung aktualisiern         & POST & /api/registrations/ & ClientId, Messaging Token & - \\
        \hline
        Registrierung entfernen         & DELETE & /api/registrations/ & ClientId & - \\
        \hline
        Relevante Registrierung finden         & Post & /api/registrations/tokens & NotificationDto & [RegistrationDto] \\
        \hline
    \end{tabular}\label{tab:registrationsapimethods}
\end{table}

\clearpage

\subsubsection*{Fachliches}

Weiter gibt es Operationen welche eine fachliche Handlung darstellen die nicht der Verwaltung von Konfigurationen dienen.
Die API für diese Operationen folgen dem folgenden Schema:

/api/entity-name/action-name

Der Cloud Service bietet die folgenden fachlichen Endpunkte an:

\begin{table}[h]
    \centering
    \begin{tabular}{|l|l|l|l|l|}
        \hline
        \textbf{Aktion} & \textbf{HTTP} & \textbf{Pfad} & \textbf{Body} & \textbf{Response} \\
        \hline
        Notifikation versenden         & POST & /api/notifications/send & NotificationDto & NotificationSendResult \\
        \hline
        Notifikation wiederholen        & POST & /api/notifications/retry & Notification Id  & NotificationSendResult \\
        \hline
    \end{tabular}\label{tab:notificationapimethods}
\end{table}

\clearpage

        \subsection{Admin UI}\label{subsec:admin-ui}

\subsubsection{Framework Grundlagen}
Auf Wunsch des Kunden wird das Admin UI als Web-Applikation auf Basis des React Admin Frameworks entwickelt.
Bei React-Admin baut auf dem React-Framework auf und vereinfacht das Erstellen, Lesen, Bearbeiten und Löschen von Datensätzen.\cite{react-admin}
Es setzt auf den folgenden Technologien auf:
\begin{itemize}
    \item React
    \item material UI
    \item React Router
    \item Redux
    \item Redux Saga
    \item React Final Form
\end{itemize}

Das Framework abstrahiert eine grosse Bandbreite an Funktionalität welche häufig in Administrations-Oberflächen gefordert sind.
Die Voraussetzung hierfür ist eine konsistente API welche für alle Routen dieselben Endpunkte anbieten~\cite{react-admin}.

\subsubsection{Anwendung}
Die konfigurierbaren Elemente werden jeweils in einer eigenen Komponente verwaltet.

\begin{figure}[h]
    \centering
    \label{fig:adminUi-packages}
    \includegraphics[width=.5\linewidth]{graphics/Admin-Ui-export}
    \caption[Admin-Ui Package Diagramm]{Admin-Ui Package Diagramm}
\end{figure}
Die Anfragen werden von jeweils von einem Data-Provider durchgeführt, der an die API des Cloudservices angepasst wird.
Vor jedem Request wird der Auth-Provider aufgerufen um zu verifizieren, dass der aktuelle Benutzer auch berechtigt ist, die gewünschte Aktion durchzuführen.

\clearpage







\subsection{Tests}

\subsubsection*{Benutzertests}
Wurden mit Daniel Jossen zusannem an der FH gemacht.
Mehr Tests waren wegen Ferienabwesenheiten auf Kundenseite nicht möglich.
Macht nicht so viel sinn mit nur einem Ipad.


\subsubsection*{Benutzertests}
Macht nicht so viel sinn mit nur einem Ipad.

\clearpage
\subsection{Lessons Learned}

\clearpage
\begin{itemize}
    \item Die Gegensprechanlage ist komplett Weggefallen.
    \subitem Grundsätzlich Stünde eine Lib. für IOS in NS zur Verfügung. Der Android Teil ist da noch "TODO"
    \item Die Rückfärbung der Buttons auf Grün ist weggefallen da der Handshake so nicht implmentiert wurde.
\end{itemize}


Hearusforderungen waren:
\begin{itemize}
    \item Dokumentation fehlt komplett in der Planung
    \item IOS Recherche viel aufwändiger als erwartet.
    \item Nativescript aufwändiger zu gebrauchen als erwartet.
    \item AWS aufsetzen ist alles andere als trivial
    \item Keine Erfahrung mit Mobile Development
    \item Stärkerer Roter Faden von Anfang an hätte geholfen
    \item Mehr testing, viel Mehr Testing.
    \item Refactoring kostet Zeit, kanns aber wert sein.
\end{itemize}

Darus mitgenommen haben wir:
\begin{itemize}
    \item Gute Planung macht sich bezahlt. (Roter Faden, Doku mit einplanen, Standortbestimmung)
    \item Gute Konzepte machen sich bezahlt.
    \item Best Practices gibt es aus einem Grund. (Nativ ist besser)
    \item DevOps ist schwer.
\end{itemize}





