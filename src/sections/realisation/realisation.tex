\section{Umsetzung}\label{sec:umsetzung}

    \subsection{Resultate}

        Das Praxisrufsystem wurde wie im Kapitel 5 - Konzept beschrieben umgesetzt.
Es wurden die drei Komponenten Mobile Client, Cloud Service und Admin UI implementiert.
Zudem wurde als Firebase Messaging als Messaging Service angebunden, um Benachrichtigungen zwischen Mobile Clients zu versenden.
Der Mobile Client kann dabei verwendet werden, um Benachrichtigungen zu versenden und empfangen.
Cloud Service und Admin UI ermöglichen die Konfiguration für Versand und Empfang von Benachrichtigungen zu verwalten und anzuwenden.
Weiter wurde mit Amazon Web Services (AWS) eine CI/CD Umgebung aufgebaut, die es erlaubt Cloud Service und das Admin UI zu betreiben und testen.
Diese Umgebung wird dem Kunden als Template dienen, wie er das Praxisrufsystem in der Praxis betreiben kann\footnote{Siehe Anhang D}.

Im Rahmen des Projektes wurden damit die Meilensteine M01 bis M06\footnote{Siehe Kapitel 2.2} erreicht.
Umgesetzt wurden die Meilensteine mit den folgenden User Stories\footnot{Siehe Kapitel 3} inklusive aller dazu definierten Features und Szenarien\footnote{Siehe Anhang E}:

\begin{itemize}
    \item U01 - Benachrichtigung versenden
    \item U02 - Benachrichtigungen empfangen
    \item U03 - Nur relevante Benachrichtigungen empfangen
    \item U04 - Auf Benachrichtigungen aufmerksam machen
    \item U05 - Verpasste Benachrichtigungen anzeigen
    \item U06 - Fehler beim Versenden von Benachrichtigungen anzeigen
    \item U07 - Konfiguration auf Mobile Client auswählen
    \item U12 - Mehrere Mobile Clients konfigurieren
    \item U13 - Individuelle Konfiguration pro Mobile Client
    \item U14 - Zentrale Konfigurationsverwaltung
    \item T01 - Mobile Client unterstützt IPads
    \item T02 - Mobile Client unterstützt Android Tablets
    \item T03 - Geteilte Code Basis für Android and IOS
    \item T04 - Betrieb mit AWS
\end{itemize}

Die Meilensteine M06 bis M10 konnten im Rahmen dieses Projektes nicht umgesetzt werden.
Dementsprechend wurden keine Anforderungen dazu umgesetzt.
Dies betrifft die User Stories U08 bis U16.

\begin{itemize}
    \item U08 - Physicher Knopf am Behandlungsstuhl
    \item U09 - Text To Speech für Benachrichtigungen
    \item U10 - Direkte Unterhaltungen zwischen Mobile Clients
    \item U11 - Gruppenunterhaltungen zwischen Mobile Clients
    \item U15 - Konfiguration von direkten Anrufen
    \item U16 - Konfiguration von Gruppenanrufen
\end{itemize}

\clearpage

        \subsection{Mobile Client}\label{subsec:mobile-client}

\subsubsection{Framework Grundlagen}
NativeScript bietet eine Abstraktion zu den nativen Plattformen Android und iOS.
Die jeweilige NativeScript Runtime erlaubt es in Javascript (oder einem entsprechenden Application Framework) Code zu schreiben,
welcher direkt für die entsprechende native Umgebung kompiliert wird~\cite{ns-core-overview}.
\begin{figure}[h]
    \centering
    \label{fig:howNSWorks}
    \includegraphics[width=0.7\textwidth]{graphics/ns-common}\caption[NativeScript-Overview]{NativeScript-Overview}\textcopyright OpenJS Foundation
\end{figure}


Die Runtime agiert als Proxy zwischen Javascript und dem jeweiligen Ökosystem.
Im Falle von iOS bedeutet dies u.A. das für alle Objective-C types ein JavaScript Prototype angeboten wird.
Dies ermöglicht es direkt mit nativen Objekten zu interagieren.
Im Umkehrschluss findet eine Typenkonversion via Marshalling Service statt\cite{ns-ios-runtime}.

\input{sections/concept/nativeScriptGuide.tex}

\subsubsection{Architektur}
\begin{figure}[h]
    \label{fig:mobileClient-packages}
    \includegraphics[width=\linewidth]{graphics/MobileClient-Architecture-export}
    \caption[Mobile-Client Package Diagramm]{Mobile-Client Package Diagramm}
\end{figure}

Der Mobile-Client wird mit modularen Komponenten aufgebaut.
Dem App-Kontext werden zwei voneinander getrennte Root-Module zur Verfügung gestellt.
Ein Modul besteht aus [1..N] Page-Modulen.
Diese Page-Module wiederum setzen sich aus eigens erstellten Komponenten und vordefinierten Komponenten des Frameworks zusammen.
Das Verhalten dieser Komponenten wird durch deren Scripts und allgemein verfügbaren Services definiert.
Der Mobile-Client wird so nach einem Objekt-orientierten Paradigma aufgebaut.

\clearpage

\begin{figure}[h]
    \label{fig:mobileClient-flow}
    \includegraphics[width=\linewidth]{graphics/MobileClient-Flow-export}
    \caption[Mobile-Client Flow Chart]{Mobile-Client Flow Chart}
\end{figure}

Das eigentliche~\emph{\nameref{fig:MobileClient-Mocks}} besteht aus dem Homescreen,
der es dem Benutzer erlaubt Nachrichten zu versenden, und der Inbox welche eingegangene Nachrichten anzeigt.
Diese Ansicht ist erst nach erfolgreicher Authentifizierung erreichbar.
Hat sich der Benutzer erfolgreich angemeldet, erhält er eine Auswahl der ihm zur verfügungstehenden Konfigurationen.
Mit der Auswahl einer dieser Konfigurationen werden die vordefinierten Notification Buttons geladen und auf dem Homescreen erstellt.

Innerhalb des App-Root Kontextes sind zwei Workflows parallel aktiv.
Zum einen können die vordefinierten Nachrichten durch Betätigen des entsprechenden Buttons versendet werden.
Die Empfänger werden durch die Rule-Engine des Cloudservices ermittelt.
Bei einem Fehlschlag wird dies dem Benutzer via Pop-Up mitgeteilt und er kann entscheiden, ob die Nachricht nochmals neu versendet werden soll.

Gleichzeitig ist immer der Listener auf Firebase aktiv.
Dieser wird auf dem Client eingegangene Nachrichten in der Inbox ablegen und ein akustisches Signal abspielen.
Wird die Meldung nicht innerhalb eines definierten Zeitraums quittiert, wird das akustische Signal wiederholt.
\clearpage

\subsubsection{User Interface}
\begin{figure}[h]
    \centering
    \begin{minipage}[b]{0.4\textwidth}
        \includegraphics[width=\textwidth]{graphics/homescreen-mockup}
        \caption{HomeScreen Mockup}
    \end{minipage}
    \hfill
    \begin{minipage}[b]{0.4\textwidth}
        \includegraphics[width=\textwidth]{graphics/mockup-received}
        \caption{Inbox Mockup}
    \end{minipage}\label{fig:MobileClient-Mocks}
\end{figure}

Die Buttons der Meldungen werden in einem 3x3 Grid auf dem Homescreen dynamisch angelegt.
Nach Betätigung eines Buttons wird dieser in Rot eingefärbt, bis eine Rückmeldung über Erfolg oder Misserfolg vom Cloudservice eingegangen ist.

Die Gegensprechfunktion ist ebenfalls auf dem Homescreen angedacht.
Hier sollten die Gesprächspartner direkt mit einem entsprechenden Button angewählt werden.
Diese Funktionalität wie beschrieben in den Userstories U10 und U11 sind im Projektverlauf ausserhalb des Umsetzungsscopes gefallen und daher nicht weiter spezifiziert worden.

Die Inbox besteht aus einer Scrollview, welche eingegangene Nachrichten als Cards darstellt.
Eine Nachricht enthält:
\begin{itemize}
    \item Titel der Nachricht
    \item Absender der Nachricht
    \item Detail Text
    \item Icon
\end{itemize}

Das Icon könnte noch dynamisch mit dem jeweiligen Nachrichtentyp verknüpft werden.
Solche wurden fachlich jedoch noch nicht definiert.

\clearpage

\subsubsection{Cloud Service}

Der Cloud Service wurde wie im Konzept beschrieben umgesetzt und an den Messaging Service angebunden.

Die API ist unter www.praxisruf.ch/api erreichbar.
www.praxisruf.ch/swagger-ui.html bietet zudem zu Testzwecken die Möglichkei die Endpoints direkt anzusprechen.

\begin{minipage}[b]{1\textwidth}
    \includegraphics[width=\textwidth]{graphics/screenshots/cloud/swagger-home}
    \caption{Home}
\end{minipage}

Sorgt neben der API auch die Authentifikation die umgesetzt wurde.

\clearpage

\subsubsection{Admin UI}

\begin{figure}[h]
    \centering
    \begin{minipage}[b]{0.4\textwidth}
        \includegraphics[width=\textwidth]{graphics/screenshots/adminui/login}
        \caption{Login}
    \end{minipage}
    \hfill
    \begin{minipage}[b]{0.4\textwidth}
        \includegraphics[width=\textwidth]{graphics/screenshots/adminui/configuration-all}
        \caption{Configuration Overview}
    \end{minipage}
    \label{fig:AdminUI-Screens1}
\end{figure}

\begin{figure}[h]
    \centering
    \begin{minipage}[b]{0.4\textwidth}
        \includegraphics[width=\textwidth]{graphics/screenshots/adminui/configuration}
        \caption{Login}
    \end{minipage}
    \hfill
    \begin{minipage}[b]{0.4\textwidth}
        \includegraphics[width=\textwidth]{graphics/screenshots/adminui/notification-type}
        \caption{Configuration Overview}
    \end{minipage}
    \label{fig:AdminUI-Screens2}
\end{figure}




Add some screen shots

\subsection{Tests}

\subsubsection*{Benutzertests}
Wurden mit Daniel Jossen zusannem an der FH gemacht.
Mehr Tests waren wegen Ferienabwesenheiten auf Kundenseite nicht möglich.
Macht nicht so viel sinn mit nur einem Ipad.


\subsubsection*{Benutzertests}
Macht nicht so viel sinn mit nur einem Ipad.

\clearpage
\subsection{Lessons Learned}

\clearpage
\begin{itemize}
    \item Die Gegensprechanlage ist komplett Weggefallen.
    \subitem Grundsätzlich Stünde eine Lib. für IOS in NS zur Verfügung. Der Android Teil ist da noch "TODO"
    \item Die Rückfärbung der Buttons auf Grün ist weggefallen da der Handshake so nicht implmentiert wurde.
\end{itemize}


Hearusforderungen waren:
\begin{itemize}
    \item Dokumentation fehlt komplett in der Planung
    \item IOS Recherche viel aufwändiger als erwartet.
    \item Nativescript aufwändiger zu gebrauchen als erwartet.
    \item AWS aufsetzen ist alles andere als trivial
    \item Keine Erfahrung mit Mobile Development
    \item Stärkerer Roter Faden von Anfang an hätte geholfen
    \item Mehr testing, viel Mehr Testing.
    \item Refactoring kostet Zeit, kanns aber wert sein.
\end{itemize}

Darus mitgenommen haben wir:
\begin{itemize}
    \item Gute Planung macht sich bezahlt. (Roter Faden, Doku mit einplanen, Standortbestimmung)
    \item Gute Konzepte machen sich bezahlt.
    \item Best Practices gibt es aus einem Grund. (Nativ ist besser)
    \item DevOps ist schwer.
\end{itemize}





