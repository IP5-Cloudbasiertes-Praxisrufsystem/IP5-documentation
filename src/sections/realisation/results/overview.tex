Das Praxisrufsystem wurde wie im Kapitel 5 - Konzept beschrieben umgesetzt.
Es wurden die drei Komponenten Mobile Client, Cloud Service und Admin UI implementiert.
Zudem wurde Firebase Messaging als Messaging Service angebunden, um Benachrichtigungen zwischen Mobile Clients zu versenden.
Der Mobile Client kann dabei verwendet werden, um Benachrichtigungen zu versenden und empfangen.
Cloud Service und Admin UI ermöglichen die Konfiguration für Versand und Empfang von Benachrichtigungen zu verwalten und anzuwenden.
Weiter wurde mit Amazon Web Services (AWS) eine CI/CD Umgebung aufgebaut, die es erlaubt Cloud Service und das Admin UI zu betreiben und testen.
Diese Umgebung wird dem Kunden als Template dienen, wie er das Praxisrufsystem in der Praxis betreiben kann\footnote{Siehe Anhang D}.

Im Rahmen des Projektes wurden damit die Meilensteine M01 bis M06\footnote{Siehe Kapitel 2.2} erreicht.
Umgesetzt wurden die Meilensteine mit den folgenden User Stories\footnote{Siehe Kapitel 3} inklusive aller dazu definierten Features und Szenarien\footnote{Siehe Anhang E}:

\begin{itemize}
    \item U01 - Benachrichtigung versenden
    \item U02 - Benachrichtigungen empfangen
    \item U03 - Nur relevante Benachrichtigungen empfangen
    \item U04 - Auf Benachrichtigungen aufmerksam machen
    \item U05 - Verpasste Benachrichtigungen anzeigen
    \item U06 - Fehler beim Versenden von Benachrichtigungen anzeigen
    \item U07 - Konfiguration auf Mobile Client auswählen
    \item U12 - Mehrere Mobile Clients konfigurieren
    \item U13 - Individuelle Konfiguration pro Mobile Client
    \item U14 - Zentrale Konfigurationsverwaltung
    \item T01 - Mobile Client unterstützt IPads
    \item T02 - Mobile Client unterstützt Android Tablets
    \item T03 - Geteilte Code Basis für Android and IOS
    \item T04 - Betrieb mit AWS
\end{itemize}

Die Umsetzung der CI/CD Pipeline sowie das Erstellen der Konzepte für den Mobile Client, haben mehr Zeit in Anspruch genommen, als geplant war.
Aufgrund dieser Verzögerungen konnten die Meilensteine M06 bis M10 nicht umgesetzt werden.
Dementsprechend wurden keine Anforderungen dazu umgesetzt.
Dies betrifft die User Stories U08 bis U16\footnote{Siehe Anhang E}:

\begin{itemize}
    \item U08 - Physicher Knopf am Behandlungsstuhl
    \item U09 - Text To Speech für Benachrichtigungen
    \item U10 - Direkte Unterhaltungen zwischen Mobile Clients
    \item U11 - Gruppenunterhaltungen zwischen Mobile Clients
    \item U15 - Konfiguration von direkten Anrufen
    \item U16 - Konfiguration von Gruppenanrufen
\end{itemize}

\clearpage
