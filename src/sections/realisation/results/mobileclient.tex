\subsubsection{Mobile Client}\label{subsec:mobile-client-realisation}

Dieses Kapitel zeigt die umgesetzten Ansichten des Mobile Clients.
Eine detaillierte Beschreibung, wie der Mobile Client bedient werden kann, befindet sich im Anhang der Projektdokumentation.

\subsubsection*{Anmeldung und Konfiguration}

Wird die Mobile Client Applikation zu ersten Mal geöffnet, muss die korrekte Konfiguration geladen werden.
So können Buttons für die benötigten Benachrichtigungen angezeigt und relevante Benachrichtigungen empfangen werden.
In einem ersten Schritt wird dem Benutzer deshalb eine einfache Login Maske angezeigt.
Darin kann sich der Benutzer mit Benutzername und Passwort anmelden.
War die Anmeldung erfolgreich werden alle Konfigurationen geladen, die dem Benutzer zur Verfügung stehen.
Die verfügbaren Konfigurationen werden dem Benutzer in einer Liste angezeigt und er wird aufgefordert, die gewünschte Konfiguration auszuwählen.
Nachdem die Auswahl erfolgt ist, wird der Benutzer zur Home Seite weitergeleitet.

\begin{figure}[h]
    \centering
    \begin{minipage}[b]{0.4\textwidth}
        \includegraphics[width=\textwidth]{graphics/screenshot-login}
        \caption{Login}
    \end{minipage}
    \hfill
    \begin{minipage}[b]{0.4\textwidth}
        \includegraphics[width=\textwidth]{graphics/screenshots/mobileclient/screenshot-select-config}
        \caption{Konfiguration}
    \end{minipage}
    \label{fig:MobileClient-Screens1}
\end{figure}

\clearpage

Im Tab Home der Startseite werden Buttons angezeigt um Benachrichtigungen zu versenden.
Die Buttons werden hier dynamisch aus der geladenen Konfiguration angezeigt.
Diese Konfiguration beinhaltet den Text der auf dem Button angezeigt wird sowie den Inhalt der Benachrichtigung versendet wird.

Klickt der Benutzer auf einen der Buttons wird die entsprechende Benachrichtigung versendet.
Die Vermittlung, wem diese Benachrichtigung zugestellt wird, liegt bei dem Cloud Service.
Dieser entscheidet Anhand der registrierten Clients und den vorhanden Konfigurationen, welche Clients Empfänger für diese Benachrichtigung sind.
Schlägt des Versenden der Benachrichtigung an mindestens einen Empfägner fehl, wird dies dem Benutzer angezeigt.
Er hat dann die Möglichkeit, das Versenden an diese Empfänger wiederholen.

\begin{figure}[h]
    \centering
    \begin{minipage}[b]{0.4\textwidth}
        \includegraphics[width=\textwidth]{graphics/screenshots/mobileclient/screenshot-homescreen}
        \caption{Home}
    \end{minipage}
    \hfill
    \begin{minipage}[b]{0.4\textwidth}
        \includegraphics[width=\textwidth]{graphics/screenshots/mobileclient/screenshots-inbox}
        \caption{Retry}
    \end{minipage}
    \label{fig:MobileClient-Screens2}
\end{figure}

\clearpage

Wurde eine Benachrichtigung empfangen, ertönt ein Audio Signal und die Benachrichtigung ist im Tab Inbox auf der Startseite ersichtlich.
Durch Klick auf einen der Einträge in der Liste, kann der Benutzer die empfangene Benachrichtigung Quittieren.
Wenn die Inbox Benachrichtigungen enthält die nicht Quittiert wurden, wiederholt der Client im Abstand von X Sekunden das Audiosignal.
Diese Quittierung erfolgt dabei nur lokal auf dem Gerät.
Der Versender wird nicht über die Quittierung benachrichtigt.

Wurde eine Benachrichtigung im Hintergrund empfangen, wird diese als Push Benachrichtigung auf dem Gerät angezeigt.
Siehe Abbildung xx.
Auch wenn die Benachrichtigung im Hintergrund empfangen wurde, wird diese in der Inbox angezeigt.

\begin{figure}[h]
    \centering
    \begin{minipage}[b]{0.4\textwidth}
        \includegraphics[width=\textwidth]{graphics/screenshots/mobileclient/screenshot-homescreen}
        \caption{Inbox}
    \end{minipage}
    \hfill
    \begin{minipage}[b]{0.4\textwidth}
        \includegraphics[width=\textwidth]{graphics/screenshots/mobileclient/screenshots-inbox}
        \caption{Push Benachrichtigung}
    \end{minipage}
    \label{fig:MobileClient-Screens3}
\end{figure}

\clearpage

