\subsubsection{Admin UI}

Mit dem Admin UI bietet das Praxisrufsystem dem Praxisverantwortlichen die Möglichkeit die Konfiguration des Systems zu verwalten.
Da nur der Praxisverantworliche das Admin UI verwenden soll, ist die Benutzeroberfläche durch ein Login geschützt.
Die entsprechenden Anmeldeinformationen müssen bei Installation des Cloud Services vom Betreiber manuell konfiguriert werden.
Das Admin UI beinhaltet vier Bereiche.

\begin{figure}[h]
    \centering
    \begin{minipage}[b]{0.4\textwidth}
        \includegraphics[width=\textwidth]{graphics/screenshots/adminui/login}
        \caption{Login}
    \end{minipage}
    \hfill
    \begin{minipage}[b]{0.4\textwidth}
        \includegraphics[width=\textwidth]{graphics/screenshots/adminui/configuration-all}
        \caption{Configuration Overview}
    \end{minipage}
    \label{fig:AdminUI-Screens1}
\end{figure}

Der Bereich \textbf{Users} dient dazu Benutzer zu erstellen, welche sich als Benutzer am Mobile Client anmelden können.

Unter \textbf{Clients} können Geräte verwaltet werden.
Jeder Client ist eindeutig einem Benutzer zugewiesen.

Im Bereich \textbf{Notification Types} können Benachrichtigungen verwaltet werden.
Hier wird konfiguriert, welchen Text der Button für diese Benachrichtigungen im Mobile Client hat und welchen Inhalt die Benachrichtigung hat, wenn sie versendet wird.

Unter \textbf{Configurations} kommen schliesslich alle Teile zusammen.
Hier kann die Konfiguration für einen Client erfasst werden.
Dabei wird die Konfiguration einem Client zugewiesen.
Die Konfiguration hat eine Liste von Notification Types, welche auf dem zugewiesenen Mobile Client als Versenden Button angezeigt werden.
Weiter beinhaltet die Konfiguration eine Liste von Regel Parametern, welche bestimmen, welche Benachrichtigungen dem zugewiesenen Client weitergeleitet werden.

\begin{figure}[h]
    \centering
    \begin{minipage}[b]{0.4\textwidth}
        \includegraphics[width=\textwidth]{graphics/screenshots/adminui/configuration}
        \caption{Login}
    \end{minipage}
    \hfill
    \begin{minipage}[b]{0.4\textwidth}
        \includegraphics[width=\textwidth]{graphics/screenshots/adminui/notification-type}
        \caption{Configuration Overview}
    \end{minipage}
    \label{fig:AdminUI-Screens2}
\end{figure}

Jeder der Bereiche im Admin UI bietet dem Benutzer die Möglichkeit, dieses Konfigurationsobjekt anzuzeigen, erstellen, bearbeiten und löschen.
Auf der Startseite jedes Bereiches wird eine Liste mit allen relevanten Einträgen angezeigt.
Über die Schlatfläche "Create" kann der Benutzer neue Elemente erfassen.
Durch einen Klick auf ein einzelnes Element, kommt der Benutzer auf eine Ansicht, wo das Element bearbeitet werden kann.

\clearpage
