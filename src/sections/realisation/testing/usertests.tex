\subsubsection*{Benutzertests}

\subsubsection*{Testablauf}

Am 21. Juli 2021 Benutzertests wurden zusammen mit dem Auftraggeber durchgeführt.
Dazu wurde der Mobile Client auf in physisches Ipad installiert.
Zusätzlich wurde eine zweite Mobile Client Instanz auf einem Emulator gestartet.
Der Cloud Service sowie das Admin UI wurden mit Amazon Webservices deployt.
Ein Firebase Messaging Service wurde angebunden.

\begin{enumerate}
    \item Client 1 im Admin UI anlegen und Benutzer 1 zuweisen.
    \item Client 2 im Admin UI anlegen und Benutzer 1 zuweisen.
    \item Einen neuen Notification Type im Admin UI anlegen
    \item Eine Client Configuration für Client 1 erstellen und darauf den erfassten Notification Type setzen.
    \item Eine Client Configuration für Client 2 erstellen und Regel Parameter erfassen, das alle Benachrichtigungen von Client 1 empfangen werden sollen.
    \item Mobile Client auf Ipad starten
    \item Auf Ipad mit Benutzer 1 anmelden.
    \item Auf Ipad Client 2 auswählen.
    \item Mobile Client auf Emulator starten.
    \item Auf Emulator mit Benutzer 1 anmelden.
    \item Auf Emulator Client 1 auswählen.
    \item Auf Emulator Benachrichtigung auslösen.
\end{enumerate}

Als Ergebnis wird erwartet, dass die Benachrichtigung auf dem physischen Ipad ankommt und angezeigt wird.
Der Test wurde zweimal durchgeführt.
Bei der ersten Durchführung war die Mobile Client Applikation auf dem Empfänger Gerät im Vordergrund geöffnet.
Bei der zweiten Durchführung wurde die Mobile Client Applikation auf dem Empfänger Gerät minimiert.
In beiden Fällen wurde erwartet, dass die Benachrichtigung in der Inbox des Empfängers angezeigt wird.
Im zweiten Fall wurde zusätzlich erwartet, dass eine Push-Benachrichtigung angezeigt wird.

Mehr oder frühere Benutzertests konnten aufgrund der pandemischen Situation leider durchgeführt werden.

\subsubsection*{Feedback vom Benutzer}

Aus den Tests mit dem Benutzer sind folgende zusätzliche Anforderungen hervorgegangen:

\begin{itemize}
    \item Wenn Benachrichtigungen eingehen sollte ein Audiosignal ertönen.
    \item Wenn Benachrichtigungen nicht quittiert werden, soll ein Erinnerungston ertönen.
    \item Wenn Benachrichtigungen quittiert werden, soll der Versender darüber informiert werden.
\end{itemize}

Um diesen Wünschen Gerecht zu werden, wurden die Szenarien S07 und S09 hinzugefügt.
Diese Anforderungen konnten auch noch umgesetzt werden.
Der dritte Wunsch des Kunden, die Quittierung von Benachrichtigungen an den Sender weitergeleitet wird konnte aus zeitlichen Gründen nicht mehr umgesetzt werden.

\clearpage

