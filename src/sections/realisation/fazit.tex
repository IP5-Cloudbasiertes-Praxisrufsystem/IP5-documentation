\subsection{Fazit}

In diesem Kapitel werden die zentralen Herausforderungen während der Projektarbeit und die Schlussfolgerungen die wir daraus ziehen beschrieben.

Die grösste Herausforderung im Projekt waren Entwicklung und Konzept des Mobile Clients mit einer geteilten Codebasis für IOS und Android.
Die Recherchen und Tests in diesem Bereich haben deutlich mehr Zeit in Anspruch genommen, als ursprünglich geplant.
Mit der gewählten Technologie Native Script konnten die Anforderungen and den Mobile Client schlussendlich umgesetzt werden.
Grosse Teile des Mobile Clients konnten für Android und IOS gleichzeitig umgesetzt werden.
An den Stellen wo die Applikation mit dem Betriebssystem interagieren muss, müssen aber Unterschiede gemacht werden und Betriebssystemspezifische Plugins verwendet werden.
Die Dokumentation des Frameworks zeigt, das es eine Vielzahl von Plugins gibt und somit fast alle Funktionen die eine native Applikation bietet auch mit Native Script umgesetzt werden können.
Während der Entwicklung mussten wir jedoch feststellen, dass diese Anbindungen in der Praxis oft nicht wie erwartet funktionieren.
Entweder, weil Plugins unvollständig dokumentiert und implementiert sind oder, weil ein Plugin nur mit einer spezifischen Version von Nativescript oder IOS funktioniert.
Dies zeigt, dass eine geteilte Codebasis für mehrere Plattformen Zeit bei der Entwicklung und Wartung sparen kann.
Sie bringt aber den Nachteil, dass die Applikation komplizierter wird, weil trotzdem oft zwischen den unterstützten Betriebssystemen unterschieden werden muss.
Weiter erschwert es die Verwendung von Betriebssystemfunktionen und Gerätehardware, da diese für alle Betriebssystem abstrahiert werden müssen.
Schlussendlich bringt dies eine zusätzliche Schicht in die Architektur die bei Änderungen am Betriebssystem und am Framework änderungen benötigt.
Dies erhöht den Wartungsaufwand deutlich und kann langfristig Probleme für die Kompatibilität mit dem Betriebssystem verursachen.

Wir schliessen daraus, dass eine geteilte Basis vor allem bei Applikationen die keine oder nur wenig Interaktion mit dem Betriebssystem benötigen, einen Vorteil bietet.
Sobald aber kritische Teile der Applikation mit Betriebssystemnahen Funktionen zusammenhängen, empfehlen wir mehrere native Applikationen zu entwickeln.
Der Nachteil, dass Funktionen doppelt implementiert werden müssen wird durch bessere Zukunftssicherheit und einfachere Anbindung an das Betriebssystem aufgehoben.

Eine weitere Herausforderung war das Konzept für den Cloud Service zu erstellen.
Dieser musste ein konfigurierbares Subscription System bieten über den die Mobile Clients Benachrichtigungen versenden und empfangen können.
Bedingung dafür war dass, individuell Regeln konfiguriert werden können die an unterschiedlichen Bedingungen prüfen, welche Benachrichtigungen für welche Clients relevant sind.
Dabei ist es wichtig, dass das Konzept es ermöglicht in Zukunft mit geringem Aufwand weitere Regeln zu implementieren.
Neben dem Regelwerk für Benachrichtigungen, war auch die Erweiterbarkeit und Skalierbarkeit des Cloud Services als Ganzes eine Herausforderung.
Anfänglich sind wir davon ausgegangen, dass sich der Cloud Service als einfache Appliaktion mit einem Endpunkt für Konfiguration und einem Endpunkt für Benachrichtigungen umsetzen lässt.
Die Entwicklungs- und Konzeptphase haben aber gezeigt, dass es auch innerhalb der Domänen eine saubere Aufteilung braucht.
Dementsprechend musste die API aufgeteilt werden, um die Weiterentwicklung und Wartbarkeit des Projektes zu gewähren.
Das Aufsetzten der Entwicklungs- und Betriebsstruktur mit Amazon Webservices hat ebenfalls deutlich mehr Zeit als geplant in Anspruch genommen.
Diese Projektarbeit hat gezeigt, das AWS alle nötigen Mittel bietet, um ein cloudbasiertes Praxisrufsystem zu betreiben.
Damit dies effizient umgesetzt werden kann ist es aber essenziell, ein Konzept zu erstellen, welches die benötigten Dienste und Konfigurationen beschreibt.
Das Installationshandbuch im Anhang kann als Vorlage für ein solches Konzept dienen\footnote{Siehe Anhang D}.

Wir haben gelernt, dass es zentral ist alle Konzepte bei Projektstart zu erstellen.
Die Konzepte müssen dabei nicht von Anfang an ausgereift sein.
Sie dienen aber als Startpunkt und Orientierungshilfe im Projekt.
Dabei ist es wichtig, dass auch Betriebsinfrastruktur, Skalierbarkait und Erweiterbarkeit von Anfang an bedacht werden.

Die letzte Herausforderung, die hier erwähnt werden soll, betrifft das Erarbeiten der Anforderungen.
Die Anforderungen und Prioritäten laufend mit dem Auftraggeber zu besprechen hat für die Projektarbeit immer klare nächste Ziele gegeben.
Da nicht alle Anforderungen bei Projektanfang vollständig definiert wurden, war es teilweise schwer den aktuellen Stand des Projektes einzuordnen.

Für weitere Projekte in diesem Rahmen empfehlen wir, alle relevanten Anforderungen bei Projektstart so detailliert wie möglich zu erarbeiten.
Diese Anforderungen können im Projektverlauf regelmässig besprochen, priorisiert und wenn nötig angepasst werden.
Dadurch ist es einfacher den Fortschritt des Projekts zu evaluieren und trotzdem möglich schnell auf neue Erkenntnisse zu reagieren.

\clearpage
