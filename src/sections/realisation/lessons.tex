\subsection{Lessons Learned}

\subsubsection*{Vollständige Konzepte}

Die Konzepte für Mobile Client und Cloud Service sind teilweise während der Entwicklung entstanden.
So hat sich z.B. die API nach aussen während des Projektes durch Refactorings geändert, weil Anpassungen nötig werden.
Im Konzept fehlten am Anfang auch ein Konkreter Plan für den Betrieb mit AWS.
Wir haben gelernt, dass es sich bezahlt macht alle Konzepte von Anfang an zu haben.
Diese Konzepte müssen dabei nicht von Anfang an ausgereift sein.
Sie Dienen als Startpunkt und Orientierungshilfe und können sich während des Projetes anpassen.
Für dieses Projekt wäre es insbesondere gut gewesen, die nötigen AWS Servives bereits in ein Konzept aufzunehmen.

\subsubsection*{Projektplanung}

Wir haben gute Erfahrungen damit gemacht, die Prioritäten vorzu mit dem Kunden abzusprechen und umzusetzen.
Rückblickend wäre es aber für diese Planung und Sitzungen hilfreich gewesen, einen Konkreteren Plan zu haben.
Die Meiolensteine die zum Anfang definiert wurden haben einen groben Überblick gegeben.
Es wäre aber hilfreich für die Umsetzung gewesen, von Anfang an schon konkretere Anforderungen zu haben.
Für dieses Projekt hätte es sich z.B. angeboten nach der Evaluierung der Technologien die Anfoderungen bereits detaillierter festzuhalten.
So hätte sich die Umsetzung effizienter gestalten lassen und man hätte mehr Features umsetzen können.

Ein weiterer Punkt bei der Projektplanung ist, dass das Schreiben des Projektberichts überhaupt nicht im Projektplan eingeplant war.
Dementsprechend war das Berichtschrieben spät dran und aufwändig.
Für künftige Projekte würden wir auf jeden Fall Zeit für  den Bericht bereits im Plan einplanen.
Und dies über die ganze Dauer des Projekts und nicht erst am Ende.

\subsubsection*{Go Native}

Eine geteilte Codebasis für mehrere Platformen kann Zeit bei der Entwicklung und Wartung sparen.
Es bringt aber den Nachteil, dass die Applikation kompliziertwer wird, weil zwischen den unterstützten Betriebssystemen unterschieden werden muss.
Weiter erschwert es die Verwendung von Betriebssystemfunktionen und Gerätehardware, da diese für alle Betriebssystem abstrahiert werden müssen.
Schlussendlich bringt es eine zusätzliche Schicht in die Architektur, bei der Kompabilität eine grosse Rolle spielen kann.
Für Applikationen die keine oder nur wenig mit direkt mit dem Betriebssystem interagieren kann eine geteilte Code Basis unter dem Strich ein Vorteil sein.
Sobald aber kritische Teile der Applikation mit Betriebssystemnahen Funktionen zusammenhängen, empfehlen wir mehrere native Applikationen zu entwickeln.
So entsteht teilweise doppelter Code.
Dieser kann aber einfacher erstellt und gewartet werden und bietet bessere Kompabilitätsgarantie.

\clearpage
