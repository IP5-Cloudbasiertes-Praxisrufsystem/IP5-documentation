\section{Anforderungen}\label{sec:anforderungen}

\subsection{User Stories}\label{subsec:user-stories}

\subsubsection*{Projekt Scope}


Nummeriert mit U01-Uxx - High Level - Fachliche Anforderung

\subsubsection*{Praxismitarbeiter}

\begin{table}[h]
    \centering
    \begin{tabular}{|l|p{13cm}|c|}
        \hline
        \textbf{Id} & \textbf{Anforderung}                                                                                                                                      & \textbf{Features} \\
        \hline
        U01         & ALS Praxismitarbeiter WILL ich relevante Anfragen versenden können, DAMIT ich über Probleme und Bedarf notifizieren kann. & F0x \\
        \hline
        U02         & ALS Praxismitarbeiter WILL ich relevante Notifikationen empfangen können, DAMIT ich über Probleme und Bedarf notifiziert werde. & F0x \\
        \hline
        U03         & ALS Praxismitarbeiter WILL will nur Notifikationen sehen die für mich relevant sind, DAMIT nicht unnötig belästigt werde. & F0x \\
        \hline
        U04         & ALS Praxismitarbeiter WILL ich über eingehende Notifikationen aufmerksam gemacht werden, DAMIT ich Notifikationen nicht übersehe. & F0x \\
        \hline
        U05         & ALS Praxismitarbeiter WILL ich sehen welche Notifikationen ich verpasst habe, DAMIT ich vergessenes nachholen kann. & F0x \\
        \hline
        U06         & ALS Praxismitarbeiter WILL ich wissen, wenn eine Notifikation nicht versendet wurde, DAMIT entsprechend reagieren kann. & F0x \\
        \hline
        U7          & ALS Praxismitarbeiter WILL ich einen Knopf an meinem Stuhl haben DAMIT ich notifikationen darüber versenden kann. & F0x \\
        \hline
        U8          & ALS Praxismitarbeiter WILL ich, dass mir Notifikationen vorgelesen werden DAMIT ich informiert werde ohne meine Arbeit zu unterbrechen & F0x \\
        \hline
        U9          & ALS Praxismitarbeiter WILL ich einen anderen Client anrufen können DAMIT Fragen direkt geklärt werden können. & F0x \\
        \hline
        U10         & ALS Praxismitarbeiter WILL ich Unterhaltungen mit mehreren anderen Clients gleichzeitig führen können DAMIT komplexe Fragen direkt geklärt werden können. & F0x \\
        \hline
    \end{tabular}\label{tab:userstories1}
\end{table}

\subsubsection*{Praxisverantwortlicher}

\begin{table}[h]
    \centering
    \begin{tabular}{|l|p{13cm}|c|}
        \hline
        \textbf{Id} & \textbf{Anforderung}                                                                                                                                                     & \textbf{Features} \\
        \hline
        U11         & ALS Praxisverantwortlicher WILL mehrere Geräte verwalten kann DAMIT jedes Gerät für Verwendungsort optimiert werden kann. & F0x \\
        \hline
        U12         & ALS Praxisverantwortlicher WILL ich definieren können welches Gerät, welche Anfragen versenden kann DAMIT jedes Gerät für Ort und Zweck optimiert werden kann & F0x \\
        \hline
        U13         & ALS Praxisverantwortlicher WILL ich die Verwaltung und Konfiguration des Praxisrufsystems zentral verwalten DAMIT overhead minimiert wird. & F0x \\
        \hline
        U14         & ALS Praxisverantwortlicher WILL ich definieren können, welche Geräte mit welchen anderen Geräten telefonieren können DAMIT meinen Mitarbeitern das Arbeiten erleichtere. & F0x \\
        \hline
    \end{tabular}\label{tab:userstories2}
\end{table}


\clearpage

\subsection{Technisch}\label{subsec:technisch}


\subsubsection*{Mobile CLient}

\begin{table}[h]
    \centering
    \begin{tabular}{|l|p{13cm}|c|}
        \hline
        \textbf{Id} & \textbf{Anforderung}                                                              & \textbf{Features} \\
        \hline
        T01         & Die Codebasis des Mobile Client MUSS für Android und IOS verwendet werden können. & F0x \\
        \hline
        T02         & Der Mobile CLient MUSS für IOS auf IPad optimiert werden.                         & F0x               \\
        \hline
        T03         & Der Mobile CLient SOLL für Android Tablets opitimiert werden.                     & F0x               \\
        \hline
    \end{tabular}\label{tab:userstories3}
\end{table}

\subsubsection*{Betrieb}
\begin{table}[h]

    \centering
    \begin{tabular}{|l|p{13cm}|c|}
        \hline
        \textbf{Id} & \textbf{Anforderung}                                                 & \textbf{Features} \\
        \hline
        T10         & Der Betrieb aller Cloud Services und Web UIs MUSS über AWS erfolgen. & F0x               \\
        \hline
    \end{tabular}\label{tab:userstories4}
\end{table}

\clearpage

\subsection{Features}\label{subsec:features}

Nummeriert mit F01-Fxx - Lower Level, näher am technischen - Spezifiziert die Anforderungen die Umgesetzt werden sollen

\subsubsection*{Projekt Scope}


\begin{table}[h]

    \centering
    \begin{tabular}{|l|p{13cm}|c|}
        \hline
        \textbf{Id} & \textbf{Feature}                                                                                                      & \textbf{User Stories} \\
        \hline
        F01         & Der Betrieb aller Cloud Services und Web UIs MUSS über AWS erfolgen.                                                  & Uxx                 \\
        \hline
        F02        & Muss einen Mobile Client geben                                                                                        & Uxx                 \\
        \hline
        F03        & Muss einen Cloud Service geben um Notifications zu pushen                                                             & Uxx                 \\
        \hline
        F04        & Muss ein Admin Web UI geben.                                                                                          & Uxx                 \\
        \hline
        F05        & Mobile Client muss Ton abspielen, wenn Notification eingeht                                                           & Uxx                 \\
        \hline
        F06        & Mobile Client muss Ton nach X Sekunden abspielen wenn Notifikation nicht quittiert wurde & Uxx \\
        \hline
        F07        & Mobile Client muss Möglichkeit bieten Notifikationen zu quittieren                                                    & Uxx                 \\
        \hline
        F08        & Praxismitarbeiter muss sich anmelden können                                                                           & Uxx                 \\
        \hline
        F09        & Praxismitarbeiter muss relevanten client auswählen können                                                             & Uxx                 \\
        \hline
        F10        & Mobile Client muss Buttons anzeigen, die vorkonfigurierte Notifikationen versenden & Uxx \\
        \hline
        F11        & Verfügbare Buttons und Notifications müssen gemäss gewähltem client angezeigt werden & Uxx \\
        \hline
        F12        & Der Mobile Client muss den Praxismitarbeiter darauf hinweisen, wenn der Versand einer Notifikation fehlgeschlagen ist & Uxx \\
        \hline
        F13        & Der Mobile Client muss es ermöglichen fehlgeschlagene sends direkt zu wiederholen & Uxx \\
        \hline
        F14        & CRUD für alle Konfigurationen müssen über ein Web UI möglich sein                                                     & Uxx                 \\
        \hline
        F15        & CRUD müssen syntaktisch korrekt sein.                                                                                 & Uxx                 \\
        \hline
        F15        & CRUD sollen semantisch korrekt sein.                                                                                 & Uxx                 \\
        \hline
        F16        & Notification Types müssen Konfigurierbar sein                                                                         & Uxx                 \\
        \hline
        F17        & Clients müssen Konfigurierbar sein                                                                                    & Uxx                 \\
        \hline
        F18        & Clients müssen Konfigurationen haben können                                                                           & Uxx                 \\
        \hline
        F19        & Pro Client muss definiert werden können, welche Notifikationen ihn interessieren & Uxx \\
        \hline
        F20        & Pro Client muss definiert werden können, welche Notifikationen versendet werden können & Uxx \\
        \hline
        F21        & Pro Client muss definiert werden können, welche Notifikationen versendet werden können & Uxx \\
        \hline
        F22        & Administrator muss sich anmelden können & Uxx \\
        \hline
        F22        & Mobile Client muss Background Push Benachrichtigungen unterstützen & Uxx \\
        \hline

    \end{tabular}\label{tab:features1}
\end{table}


\subsubsection*{Out Of Scope}

Weitere Features wurden noch nicht konkreter definiert. Da jeweils je "Sprint" die als nächstes umzusetzenden Features definiert wurden.

\begin{itemize}
    \item Soll Client für Raspberry Pi geben
    \item Client Configuration für einzelnen Button soll möglich sein
    \item Mobile Client soll 1:1 Unterhaltungen unterstützen
    \item Mobile Client soll 1:n Unterhaltungen unterstützen.
\end{itemize}


