\section{Anforderungen}\label{sec:anforderungen}

Die im Rahmen des Projektes umzusetzenden Anforderungen wurden während des Projektes iterativ zusammen mit dem Kunden erarbeitet.
Alle Anforderungen werden zuerst aus Fachlicher sicht mit User Stories festgehalten, die ein konkretes Bedürfnis der Benutzer beschreiben.
Weiter werden User Stories aus Sicht des Kunden festhalten, welche Rahmenbedingungen und Bedürfnisse des Auftraggebers festhalten.
Aufgrund der User Stories werden anschliessend Features definiert, welche umsetzbare Anforderungen an die Systemkomponenten definieren.

\subsection{User Stories}\label{subsec:user-stories}

\subsubsection*{Praxismitarbeiter}

\begin{table}[h]
    \centering
    \begin{tabular}{|l|p{13cm}|c|}
        \hline
        \textbf{Id} & \textbf{Anforderung}                                                                                                                                      & \textbf{Features} \\
        \hline
        U01         & Als Praxismitarbeiter möchte ich Benachrichtigungen versenden können, damit ich andere Mitarbeiter über Probleme und Anfragen informieren kann. & F0x \\
        \hline
        U02         & Als Praxismitarbeiter möchte ich Benachrichtigungen empfangen können, damit ich auf Probleme und Anfragen anderer Mitarbeiter reagieren kann. & F0x \\
        \hline
        U03         & Als Praxismitarbeiter möchte ich nur Benachrichtigungen sehen, die für mich relevant sind, damit ich meine Arbeit effizient gestalten kann. & F0x \\
        \hline
        U04         & Als Praxismitarbeiter möchte ich über empfangene Benachrichtigungen aufmerksam gemacht werden, damit ich keine Benachrichtigungen verpasse. & F0x \\
        \hline
        U05         & Als Praxismitarbeiter möchte ich sehen welche Benachrichtigungen ich verpasst habe, damit ich auf verpasste Benachrichtigungen reagieren kann. & F0x \\
        \hline
        U06         & Als Praxismitarbeiter möchte ich eine Rückmeldung erhalten, wenn eine Benachrichtigung nicht versendet werden kann, damit Benachrichtigungen nicht verloren gehen. & F0x \\
        \hline
        U07         & Als Praxismitarbeiter möchte ich auswählen können an welchem Gerät ich das Praxisrufsystem verwende und die dafür erstellte Konfiguration erhalten, damit das Praxisrufsystem optimal verwendet werden kann. & F0x \\
        \hline
        U8          & Als Praxismitarbeiter möchte ich einen physischen Knopf am Behandlungsstuhl haben damit ich notifikationen darüber versenden kann. & F0x \\
        \hline
        U9          & Als Praxismitarbeiter möchte ich, dass mir Benachrichtigungen vorgelesen werden, damit ich informiert werde, ohne meine Arbeit unterbrechen zu müssen. & F0x \\
        \hline
        U10          & Als Praxismitarbeiter möchte ich einen anderen Client anrufen können damit Fragen direkt geklärt werden können. & F0x \\
        \hline
        U11         & Als Praxismitarbeiter möchte ich Unterhaltungen mit mehreren anderen Clients gleichzeitig führen können damit komplexe Fragen direkt geklärt werden können. & F0x \\
        \hline
    \end{tabular}\label{tab:userstories1}
\end{table}

\subsubsection*{Praxisverantwortlicher}

\begin{table}[h]
    \centering
    \begin{tabular}{|l|p{13cm}|c|}
        \hline
        \textbf{Id} & \textbf{Anforderung}                                                                                                                                                     & \textbf{Features} \\
        \hline
        U12         & Als Praxisverantwortlicher möchte ich mehrere Geräte verwalten können, damit jedes Gerät für Verwendungsort optimiert ist. & F0x \\
        \hline
        U13         & Als Praxisverantwortlicher möchte ich definieren können welches Gerät, welche Anfragen versenden kann, damit jedes Gerät Verwendungsort optimiert ist. & F0x \\
        \hline
        U14         & Als Praxisverantwortlicher möchte ich die Konfiguration des Praxisrufsystems zentral verwalten können, damit das Praxisrufsystem für die Anwender optimiert werden kann. & F0x \\
        \hline
        U15         & Als Praxisverantwortlicher möchte ich definieren können, welche Geräte mit welchen anderen Geräten telefonieren können damit meinen Mitarbeitern das Arbeiten erleichtere. & F0x \\
        \hline
        U16         & Als Praxisverantwortlicher möchte ich definieren können, welche Benachrichtigung über einen physischen Knopf am Behandlungsstuhl versendet wird damit der Knopf für den Mitarbeiter optimiert ist. & F0x \\
        \hline
    \end{tabular}\label{tab:userstories2}
\end{table}


\clearpage

\subsection{Rahmenbedingungen}\label{subsec:rahmenbedingungen}

\begin{table}[h]
    \centering
    \begin{tabular}{|l|p{13cm}|c|}
        \hline
        \textbf{Id} & \textbf{Anforderung}                                                              & \textbf{Features} \\
        \hline
        T01         & Als Auftraggeber möchte ich, dass das Praxisrufsystem über IPads bedient werden kann, damit ich von bestehender Infrastruktur profitieren kann. & F0x \\
        \hline
        T02         & Als Auftraggeber möchte ich, dass das Praxisrufsystem über Android Tablets bedient werden kann, damit es in Zukunft für eine weitere Zielgruppe verwendet werden kann. & F0x \\
        \hline
        T03         & Als Auftraggeber möchte ich, dass die Codebasis für das Praxisrufsystem für Android und IOS verwendet werden kann, damit ich die Weiterentwicklung optimieren kann. & F0x \\
        \hline
        T04         & Als Auftraggeber möchte ich, dass wo möglich der Betrieb von Serverseitigen Dienstleistungen über AWS betrieben wird, damit ich von bestehender Infrastruktur und Erfahrung profitieren kann. & F0x               \\
        \hline
    \end{tabular}\label{tab:userstories3}
\end{table}


\clearpage

\subsection{Features}\label{subsec:features}

Nummeriert mit F01-Fxx - Lower Level, näher am technischen - Spezifiziert die Anforderungen die Umgesetzt werden sollen

\subsubsection*{Projekt Scope}


\begin{table}[h]

    \centering
    \begin{tabular}{|l|p{13cm}|c|}
        \hline
        \textbf{Id} & \textbf{Feature}                                                                                                      & \textbf{User Stories} \\
        \hline
        F01        & Der Betrieb aller Cloud Services und Web UIs muss über AWS erfolgen.                                                  & Uxx                 \\
        \hline
        F02        & Das Praxisrufsystem muss über einen Mobile Client auf IPads und Android Tables verwendbar sein.                                                                                        & Uxx                 \\
        \hline
        F03        & Benachrichtigungen müssen über einen zentralen Cloud Service versendet werden.                                                              & Uxx                 \\
        \hline
        F04        & Das Praxisrufsystem muss über einen zentralen Cloud Service verwaltet werden können.                                                                                           & Uxx                 \\
        \hline
        F05        & Der Mobile Client muss Ton abspielen, wenn Notification eingeht                                                           & Uxx                 \\
        \hline
        F06        & Mobile Client muss Ton nach X Sekunden abspielen wenn Notifikation nicht quittiert wurde & Uxx \\
        \hline
        F07        & Mobile Client muss Möglichkeit bieten Notifikationen zu quittieren                                                    & Uxx                 \\
        \hline
        F08        & Praxismitarbeiter muss sich anmelden können                                                                           & Uxx                 \\
        \hline
        F09        & Praxismitarbeiter muss relevanten client auswählen können                                                             & Uxx                 \\
        \hline
        F10        & Mobile Client muss Buttons anzeigen, die vorkonfigurierte Notifikationen versenden & Uxx \\
        \hline
        F11        & Verfügbare Buttons und Notifications müssen gemäss gewähltem client angezeigt werden & Uxx \\
        \hline
        F12        & Der Mobile Client muss den Praxismitarbeiter darauf hinweisen, wenn der Versand einer Notifikation fehlgeschlagen ist & Uxx \\
        \hline
        F13        & Der Mobile Client muss es ermöglichen fehlgeschlagene sends direkt zu wiederholen & Uxx \\
        \hline
        F14        & CRUD für alle Konfigurationen müssen über ein Web UI möglich sein                                                     & Uxx                 \\
        \hline
        F15        & CRUD müssen syntaktisch korrekt sein.                                                                                 & Uxx                 \\
        \hline
        F15        & CRUD sollen semantisch korrekt sein.                                                                                 & Uxx                 \\
        \hline
        F16        & Notification Types müssen Konfigurierbar sein                                                                         & Uxx                 \\
        \hline
        F17        & Clients müssen Konfigurierbar sein                                                                                    & Uxx                 \\
        \hline
        F18        & Clients müssen Konfigurationen haben können                                                                           & Uxx                 \\
        \hline
        F19        & Pro Client muss definiert werden können, welche Notifikationen ihn interessieren & Uxx \\
        \hline
        F20        & Pro Client muss definiert werden können, welche Notifikationen versendet werden können & Uxx \\
        \hline
        F21        & Pro Client muss definiert werden können, welche Notifikationen versendet werden können & Uxx \\
        \hline
        F22        & Administrator muss sich anmelden können & Uxx \\
        \hline
        F22        & Mobile Client muss Background Push Benachrichtigungen unterstützen & Uxx \\
        \hline

    \end{tabular}\label{tab:features1}
\end{table}


\subsubsection*{Out Of Scope}

Weitere Features wurden noch nicht konkreter definiert. Da jeweils je "Sprint" die als nächstes umzusetzenden Features definiert wurden.

\begin{itemize}
    \item Soll Client für Raspberry Pi geben
    \item Client Configuration für einzelnen Button soll möglich sein
    \item Mobile Client soll 1:1 Unterhaltungen unterstützen
    \item Mobile Client soll 1:n Unterhaltungen unterstützen.
\end{itemize}


