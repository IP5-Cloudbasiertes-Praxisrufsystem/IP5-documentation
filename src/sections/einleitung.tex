\section{Einleitung}

\subsubsection*{Ausgangslage}

Ärzte und Zahnärzte haben den Anspruch in Ihren Praxen ein Rufsystem einzusetzen.
Dieses Rufsystem ermöglicht, dass der behandelnde Arzt über einen Knopfdruck Hilfe anfordern oder Behandlungsmaterial bestellen kann.
Zusätzlich bieten die meisten Rufsysteme die Möglichkeit eine Gegensprechfunktion zu integrieren.
Ein durchgeführte Marktanalyse hat gezeigt, dass die meisten auf dem Markt kommerziell erhältlichen Rufsysteme auf proprietären Standards beruhen und ein veraltetes Bussystem oder analoge Funktechnologie zur Signalübermittlung einsetzen.
Weiter können diese Systeme nicht in ein TCP/IP-Netzwerk integriert werden und über eine API extern angesteuert werden.

\subsubsection*{Ziel der Arbeit}

Im Rahmen dieser Arbeit soll ein Cloudbasiertes Praxisrufsystem entwickelt werden.
Pro Behandlungszimmer wird ein Android oder IOS basiertes Tablet installiert. Auf diese Tablet kann die zu entwickelnde App installiert und betrieben werden. Die App deckt dabei die folgenden Ziele ab:

\begin{itemize}
    \item Definition und Entwicklung einer skalierbaren Softwarearchitektur
    \item App ist auf Android und IOS basierten Tablets einsetzbar
    \item App besitzt eine integrierte Gegesprechfunktion (1:1 oder 1:m Kommunikaion)
    \item Auf der App können beliebige Buttons konfiguriert werden, die anschliessend auf den anderen Tablets einen Alarm oder eine Meldung (Text- oder Sprachmeldung) generieren
    \item Textnachrichten können als Sprachnachricht (Text to Speech) ausgegeben werden
    \item Verschlüsselte Übertragung aller Meldungen zwischen den einzelnen Stationen
    \item Integration der Lösung in den Zahnarztbehandlungsstuhl über einen Raspberry PI
    \item Offene API für Integration in Praxisadministrationssystem
\end{itemize}

\subsubsection*{Problemstellung}

Die Hauptproblemstellung dieser Arbeit ist die sichere und effiziente Übertragung von Sprach- und Textmeldungen zwischen den einzelnen Tablets.
Dabei soll es möglich sein, dass die App einen Unicast, Broadcast und Mutlicast Übertragung der Daten ermöglicht.
Über eine offene Systemarchitektur müssen die Kommunikationsbuttons in der App frei konfiguriert und parametrisiert werden können.


\clearpage
\subsubsection*{Methodik}

Das Projekt IP5 Cloudbasiertes Praxisrufsystem wurde im FS21 gestartet.
Die Organisation und Kommunikation des Projektes mussten dementsprechend für die Einschränkungen wegen Corona angepasst werden.
Um sicherzustellen, dass die Kommunikation über die gesamte Projektdauer funktionieren kann, haben wir uns deshalb von Anfang an entschieden die Kommunikation über Remote- und Online Tools zu organisieren.
Für Besprechungen und Planungen wurde Microsoft Teams gewählt.
Die entsprechende Infrastruktur wurde von der FHNW zur Verfügung gestellt.

Das Projekt Cloudbasiertes Praxisrufsystem soll in vier Phasen umgesetzt werden.
In der ersten Phase wird die Organisation innerhalb des Projekt teams geklärt.
Es werden die Top Level Anforderungen besprochen und priorisiert.
Dazu werden oberflächliche Konzepte erarbeitet.
In der zweiten Phase sollen Technologien evaluiert werden die verwendet werden um die wichtigsten Anforderungen umzusetzen.
Es wird ein Proof Of Concept implementiert, der beweist, dass die technischen Voraussetzungen gegeben sind, diese Anforderungen umzusetzen.
In der dritten Phase wird auf Basis des Proof Of Concept das Praxisrufsystem umgesetzt.
Die umzusetzenden Anforderungen sind dabei in einem iterativen Verfahren zusammen mit dem Kunden zu spezifizieren.
Die aktuelle Lage wird in Regelmässigen Absätnden mit dem Kunden besprochen und die nächsten Schritte werden geklärt.

\subsubsection*{Konzepterläuterung}

Im nachfolgenden Hauptteil wird die Arbeit im Detail vorgestellt.
Der Hauptteil gliedert sich im fünf Teile.
Zuerst werden im Kapitel 2 Vorgehensweise der Projektplan und die Organisation im vorgestellt.
Anschliessend zeigt das Kapitel 3 Anforderungen, welche Anforderungen mit dem Kunden erarbeitet und umgesetzt wurden.
Das Kapitel 4 Evaluation Technologien stellt die Recherchen vor die gemacht wurden um die für die Anforderungen geeigneten Technologien zu identifizieren.
Das Kapitel 5 Konzept stellt schliesslich das deteillierte technische Konzept vor, das dem umgesetzten System zugrunde liegt.
Die Resultate der Arbeit werden dann im Kapitel 6 Umsetzung vorgstellt.


\clearpage
