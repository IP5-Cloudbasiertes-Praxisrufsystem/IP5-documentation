\section*{Zusammenfassung}

Ärzte und Zahnärzte haben den Anspruch in Ihren Praxen ein Rufsystem einzusetzen.
Dieses Rufsystem ermöglicht, dass der behandelnde Arzt über einen Knopfdruck Hilfe anfordern oder Behandlungsmaterial bestellen kann.
Heute kommerziell erhältlichen Rufsysteme beruhen meistens auf proprietären Standards und setzen veraltete Technologie zur Signalübermittlung ein. \cite{TODO: Add Citation!}
Die Problemstellung für dieses Projekt besteht darin, ein modernes, leicht konfigurierbares Praxisrufsystem umzusetzen.
Das Rufsystem soll mit Mobilen Endgeräten (IPads und Android Tablets) bedient werden können und auf modernen cloudbasierten Technologien aufbauen.
In diesem Projekt wurde ein Konzept erarbeitet, wie ein solches Rufsystem zum Versenden von Benachrichtigungen auf Knopfdruck umgesetzt werden kann.
Dieses Konzept wurde mit drei Applikationen umgesetzt.
Eine Mobile Applikation die auf dem Endgerät in der Praxis verwendet wird, eine Web Applikation zur Konfiguration des Rufsystens
und ein cloudbasierter Backend Service, welcher die Konfiguration des Systems persistent verwaltet und zwischen den Anfragen zwischen den Endgeräten vermittelt.
Um die Endgeräte über eingehende Benachrichtigungen zu informieren wurde mit Firebase Messaging ein externer Messaging Service angebunden.
Das Umgesetzte System bietet nun die Möglichkeit ein Praxisrufsystem mit mehreren Mobilgeräten als Endgeräte zu betreiben.
Es kann pro Endgerät konfiguriert werden, welche Benachrichtigungen das Gerät versenden kann und jedes Gerät zeigt genau nur die Benachrichtigungen an, die es versenden kann.
Weiter kann pro Endgerät definiert werden, welche Benachrichtigungen das Gerät empfangen möchte.
Welche Benachrichtigungen relevant sind, kann entweder anhand des Senders oder des Typs der Benachrichtigung eingestellt werden.



\clearpage
\section*{Management Summary}

Ärzte und Zahnärzte haben den Anspruch in Ihren Praxen ein Rufsystem einzusetzen.
Dieses Rufsystem ermöglicht, dass der behandelnde Arzt über einen Knopfdruck Hilfe anfordern oder Behandlungsmaterial bestellen kann.
Heute kommerziell erhältlichen Rufsysteme beruhen meistens auf proprietären Standards und setzen veraltete Technologie zur Signalübermittlung ein. \cite{TODO: Add Citation!}
Die Problemstellung für dieses Projekt besteht darin, ein modernes, leicht konfigurierbares Praxisrufsystem umzusetzen.
Das System soll zudem leicht erweiterbar sein.

In diesem Projekt wurde ein Konzept erarbeitet, wie ein solches Rufsystem zum Versenden von Benachrichtigungen auf Knopfdruck umgesetzt werden kann.
Das umgesetzte Praxisrufsystem wird vom Endbenutzer auf einer Mobilem Applikation benutzt.
Vor der verwendung kann und muss das System konfiguriert werden.
Es muss pro Endgerät definiert werden, welche Benachrichtigungen es versenden kann und welche Benachrichtigungen es empfangen soll.

Nicht umgesetzt wurden die Funktion Benachrichtigungen vorzulesen (Text To Speech) und Gegensprechanlage.
Die Kommunikation zwischen den Endgeräten funktioniert ausschliesslich über konfigurierbare Benachrichtigungen.

Das Endprodukt das der Praxismitarbeiter verwenden wird ist eine Mobile Applikation.
Diese kann für Android und für IOs verwendet werden.
Die technische Grundlage dafür ist eine einzelne Applikation, die für die jeweilige Plattform separat gebaut wird.

Für das weitere Vorgehen sehen wir die folgenden 3 Optionen:


\textbf{1. Praxisruf einsetzten}

Das in diesem Projekt umgesetzte System ist grundsätzlich einsatzbereit.
Es kann als Praxisrufsystem eingesetzt werden, wenn nur die Funktion konfigurierbare Benachrichtigungen zu versenden benötigt wird.
Wir würden allerdings von dieser Option abraten.
Das Abonnieren von Benachrichtigungen kann bereits jetzt eingesetzt werden.
In der Praxis können aber durchaus kompliziertere Regeln benötigt werden, als das System heute unterstützt.
Weiter ist die umgesetzte Benutzeroberfläche funktional komplett.
Im Bereich Benutzerinterface Design und Usability konnte im Rahmen dieses Projektes allerdings nur wenig Arbeit gemacht werden.
Bevor Praxisruf als Produkt kommerzialisiert wird würden wir empfehlen das UI entsprechend zu optimieren.
Der grösste Punkt der gegen Option 1 spricht ist allerdings der aktuelle Funktionsumfang von Praxisruf.

\textbf{2. Praxisruf weiterentwickeln}

Als zweite Option sehen wir, das System weiterzuentwickeln.
Dies beinhaltet die Erweiterung der Konfiguration für die Abonnierung von Benachrichtigungen
sowie das Einbauen von weiteren Funktionen wie Text To Speech und der Gegensprechanlage.
Aber auch die Weiterentwicklung des Mobile Clients um dem Endbenutzer ein ansprechendes Benutzererlebniss zu bieten.

Der für dieses Projekt vorgegebene Ansatz einer geteilten Codebasis für IOS und Android brint für diese Weiterentwicklung Hindernisse mit sich.
Design und Weiterentwicklung der Benutzeroberfläche würden mit den aktuell eingesetzten Technologien problemlos funktionieren.
Die Erweiterung um die Funktionalität der Gegensprechanalge könnte hier aber problematisch werden.
Grundsätzlich unterstützten die gewählten Technologien diese Funktion zum grössten Teil.
Die Technologien sind in diesen Punkten aber noch in den Kinderschuhen und wenig ausgereift.
Viele der Komponenten die dazu verwendet werden müssten sind noch unvollständig implementiert oder haben grosse Einschränkungen bezüglich Kompatibilität.

\clearpage
\textbf{3. Praxisruf weiterentwickeln (nativ) }

Als dritte Option empfehlen wir das System wie in Option 2 beschrieben weiterzuentwickeln.
Dabei aber den bestehenden Mobile Client neu als native Applikation zu entwickeln.
Dies wird im ersten Moment mehr Aufwand bedeuten.
Es bringt aber den Vorteil von besserer Komaptibilität.
Solwohl für die bestehenden Funktionalitäten in zukünftigen Versionen der Platformen und insbesondere für die Zugriffe die für die Umsetzung einer Gegensprechanalge zu verwenden sind.
Wir gehen davon aus, das die Wartung für zwei native Applikationen am Ende nicht grösser sein wird als die Wartung eines geteilten Clients.

\textbf{Empfehlung}

Wir empfehlen für das Weitere Vorgehen die Option 3 zu wählen.
Wie in der Aufgabenstellung erwähnt, ist die Gegensprechanalge of Teil eines Praxisrufsystem.
Damit sich das System eine möglichst grossen Vorteil gegenüber bestehenden Systemen bietet würden wir unbedingt empfehlen das System zuerst weiterzuentwickelnd und um die Gegensprechfunktion zu erweitern.
Für bessere Kompatibilität und Zukunftssicherheit emfpehlen wir, bei dieser Weiterentwicklung den Mobile Client neu als native Applikation zu entwickeln.


\clearpage


\clearpage
