\begin{abstract}
Das Abstract ist eine Art Zusammenfassung des ganzen Dokuments. Es gibt einen Einblick in die Aufgabenstellung, wie diese umgesetzt wurde und welches Ergebnis erreicht wurde. Aus diesem Grund wird das Abstract immer ganz am Schluss der Arbeit verfasst. Es besteht aus einem zusammengehörenden Absatz und umfasst ungefähr 10 bis 20 Zeilen.
Formeln, Referenzen oder andere Unterbrechungen haben im Text nichts zu suchen.
Direkt unter dem Abstract folgt eine Liste von drei bis vier Stichworten/Keywords. Diese werden in alphabetischer Reihenfolge aufgelistet und beschreiben das Themengebiet der Arbeit.

\vspace{2ex}

\textbf{Keywords: Anleitung, LaTeX, Thesis, Vorlage}

\vspace{2ex}

\textbf{Management Summary} siehe PF-IK.

\end{abstract}	

\clearpage

\section*{Vorwort}

\lipsum[1-2]

Fakultativ, siehe PF-IK (URL)
